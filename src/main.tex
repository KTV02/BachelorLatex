%Preamble
\documentclass[11pt, a4paper,onecolumn ,titlepage]{article}
%\documentclass[man,floatsintext]{apa7}



%Packages
\usepackage[utf8]{inputenc}
\usepackage[T1]{fontenc}
\usepackage{lmodern}
\usepackage{amsmath,amssymb,amstext}
\usepackage{graphicx}
\usepackage{xstring}
\usepackage{suffix}
\usepackage[nohyperlinks, printonlyused, withpage, smaller]{acronym}
\usepackage{hyperref}
\usepackage{tabularx}
\usepackage{etoolbox}
\usepackage[nooneline]{caption}
\usepackage{float}
\usepackage{pdfpages}
\usepackage{microtype}
\usepackage{booktabs}
\usepackage{abstract}
\usepackage{tablefootnote}
\usepackage{blkarray}
\usepackage{shorttoc}
\usepackage[shortlabels]{enumitem}
\usepackage{threeparttable}
\usepackage{arabicfnt}
\usepackage{tocbibind}

%Einbindung der Referenzen in das Dokument via BiblaTeX
\usepackage[english]{babel}
\usepackage[style=english,english=quotes]{csquotes}
\usepackage[style=apa,backend=biber]{biblatex}
\DeclareLanguageMapping{american}{american-apa}
\addbibresource{main.bib}


%Zum Einfügen von Code
\usepackage{listings}
\usepackage{color}
\definecolor{dkgreen}{rgb}{0,0.6,0}
\definecolor{gray}{rgb}{0.5,0.5,0.5}
\definecolor{mauve}{rgb}{0.58,0,0.82}
\lstset{frame=tb,
    language=python,
    aboveskip=3mm,
    belowskip=3mm,
    showstringspaces=false,
    columns=flexible,
    basicstyle={\small\ttfamily},
    numbers=none,
    numberstyle=\tiny\color{gray},
    keywordstyle=\color{blue},
    commentstyle=\color{dkgreen},
    stringstyle=\color{mauve},
    breaklines=true,
    breakatwhitespace=true,
    tabsize=2
    numbers=left
}


%Document
\begin{document}

%Titelseite
    \begin{titlepage}
        \begin{center}

            \vspace*{1cm}

            {\large \textbf{Entwicklung eines TryHackMe Raumes}}

            \vspace{0.5cm}
            zur Sicherheitsschulung von Schwachstellen - aufgezeigt am Lernmanagementsystem \ac{ilias}

            \vspace{1.5cm}

            \textbf{Chris Benz} \\
            \small{chbenz@stud.hs-heilbronn.de}
            \\
            \vspace{0.2cm}
            \textbf{Umut Mehmet Eke}\\
            \small{ueke@stud.hs-heilbronn.de}
            \\
            \vspace{0.2cm}
            \textbf{Lennart Kremp} \\
            \small{lkremp@stud.hs-heilbronn.de}
            \\
            \vspace{0.2cm}
            \textbf{Frederik Spieß}\\
            \small{frspiess@stud.hs-heilbronn.de}


            \vfill

            \begin{figure}[H]
                \centering
                \begin{minipage}[b]{.13\linewidth} % [b] => Ausrichtung an \caption
                    \includegraphics[width=\linewidth]{author_pictures/chris_2}
                \end{minipage}\label{fig:chris}
                \hspace{.005\linewidth}% Abstand zwischen Bilder
                \begin{minipage}[b]{.13\linewidth} % [b] => Ausrichtung an \caption
                    \includegraphics[width=\linewidth]{author_pictures/umut}
                \end{minipage}\label{fig:umut}
                \hspace{.005\linewidth}% Abstand zwischen Bilder
                \begin{minipage}[b]{.13\linewidth} % [b] => Ausrichtung an \caption
                    \includegraphics[width=\linewidth]{author_pictures/lennart}
                \end{minipage}\label{fig:lennart}
                \hspace{.005\linewidth}% Abstand zwischen Bilder
                \begin{minipage}[b]{.13\linewidth} % [b] => Ausrichtung an \caption
                    \includegraphics[width=\linewidth]{author_pictures/Fred}
                \end{minipage}\label{fig:frederik}
            \end{figure}


            \vfill

            Durchführung im Rahmen der Veranstaltung \\ \glqq Praktikum sichere Software-Entwicklung\grqq\ an der Hochschule Heilbronn

            \vspace{0.2cm}

            Beaufsichtigt durch Prof. Dr.-Ing. Andreas Mayer

            \vspace{1.0cm}


            \vspace{0.8cm}

            Fakultät für Informatik\\
            Hochschule Heilbronn\\
            \today{}, Heilbronn

        \end{center}

    \end{titlepage}

%%%%%%%%%%%%%%%%%%%%%%%%%%%%%%%%%%%%%%%%%%%%%%%%%%%%%%%%%%%%%%%%%%%%%%%%%%%%%%%%%
%	Abstract
%%%%%%%%%%%%%%%%%%%%%%%%%%%%%%%%%%%%%%%%%%%%%%%%%%%%%%%%%%%%%%%%%%%%%%%%%%%%%%%%%

    \renewcommand*\abstractname{\flushleft\textbf{Abstract}\hfill}
    \begin{abstract}
        \hline
        \vspace{0.5cm}
        \noindent
        \textbf{Hintergrund}:  In einer immer stärker miteinander vernetzten Welt wird eines immer wichtiger, die Sicherheit und der Schutz digitaler Infrastruktur.
        Dies zeigt auch eine immer weiter zunehmenden Anzahl an Cyberangriffen, die es nötig macht, seine IT-Infrastruktur immer im Blick zu behalten und bestmöglich vor ungewollten Angriffen zu schützen.
        Geschultes IT-Personal mit fundierten Kenntnissen im Umgang mit Schwachstelle ist somit unerlässlich.

        \vspace{0.5cm}
        \noindent
        \textbf{Zielsetzung}: Das Ziel dieses Projekts ist das Erstellen eines TryHackMe Raumes, der anhand einer ansprechenden Storyline
        Wissen im Bereich IT-Sicherheit vermittelt und vorhandene Kenntnisse auf die Probe stellt.
        Es gilt demnach, geeignete Schwachstellen (in \ac{ilias}) zu bestimmen und eine Storyline rund um diese zu entwerfen.
        Passende Technologien sind auszuwählen und die \ac{vm}, zur späteren Einbindung in den TryHackMe Raum, auszuarbeiten.
        Abschließen gilt es alle Komponenten in einem ansprechend designten TryHackMe Raum zu vereinen.


        \vspace{0.5cm}
        \noindent
        \textbf{Ergebnisse}: Insgesamt wurden zwei \ac{vm}s mit den \ac{ilias} Versionen 5.2.3 und 5.2.4 erstellt.
        Eine Storyline rund um den Studenten TimDerHackerBoy führt die Nutzer des Raumes durch die beiden Maschinen.
        Um den Raum erfolgreich abzuschließen sind unter anderem verschiedene Schwachstellen auszunutzen, drei Flags zu finden und den erfolgreichen Umgang mit verschiedenen Tools wie Hydra unter Beweis zustellen.

        \vspace{0.5cm}
        \hline
        \vspace{1cm}
        \noindent
        \keywords{Ilias, Fehlkonfiguration, ImageMagick, Privilage Escalation, XSS}

        \vspace{2cm}
        \noindent
        Für Zugang zum Git-Repository oder bei Fragen und Anregungen melden Sie sich bitte unter den angegebene Mail-Adressen der Autorenschaft.

        \vspace{1cm}
        \noindent
        Hier geht es zum \href{https://tryhackme.com/room/t1mth3h4ck3rb0y}{\textbf{TryHackMe Raum}}.
    \end{abstract}

%%%%%%%%%%%%%%%%%%%%%%%%%%%%%%%%%%%%%%%%%%%%%%%%%%%%%%%%%%%%%%%%%%%%%%%%%%%%%%%%%
%	1.Inhaltsverzeichnis
%%%%%%%%%%%%%%%%%%%%%%%%%%%%%%%%%%%%%%%%%%%%%%%%%%%%%%%%%%%%%%%%%%%%%%%%%%%%%%%%%


     \shorttoc{Inhaltsübersicht}{1} %Inhaltübersicht ohne Unterpunkte
     \pagebreak
     \tableofcontents
     \vfill
     \pagebreak


%%%%%%%%%%%%%%%%%%%%%%%%%%%%%%%%%%%%%%%%%%%%%%%%%%%%%%%%%%%%%%%%%%%%%%%%%%%%%%%%%
%	1.Einleitung
%%%%%%%%%%%%%%%%%%%%%%%%%%%%%%%%%%%%%%%%%%%%%%%%%%%%%%%%%%%%%%%%%%%%%%%%%%%%%%%%%
    \fill
    \newpage
    \section{Einleitung}
    \label{sec:einleitung}

    \subsection{Gegenstand und Motivation}
    \label{subsec:gegenstand-motivation}
    \footnote{In dieser Arbeit wird aus Gründen der besseren Lesbarkeit das generische Maskulinum verwendet. Weibliche und anderweitige Geschlechteridentitäten werden dabei ausdrücklich mitgemeint.}
    Die Zuverlässigkeit moderner IT-Systeme entscheidet über Erfolg oder Misserfolg vieler Firmen, Branchen oder ganzer Nationen.
    In unserer immer stärker digitalisierten Welt, wird die Sicherheit und der Schutz digitaler Infrastruktur dementsprechend immer wichtiger.
    Dies zeigt auch eine immer weiter zunehmende Anzahl an Cyberangriffen, die es nötig macht, seine IT-Infrastruktur immer im Blick zu behalten und bestmöglich vor ungewollten Angriffen zu schützen.
    Geschultes IT-Personal mit fundierten Kenntnissen im Umgang mit Schwachstelle ist somit unerlässlich.
    Eine Möglichkeit, ein solches Wissen zu erlangen, bietet die Lernplattform \href{https://tryhackme.com/}{TryHackMe}.
    Sie bietet interessierten Personen die Möglichkeit, sich in Lern- und Challengeräumen umfangreiches Wissen im Bereich Ethical Hacking anzueignen.
    Und das erworbene Wissen in einer realitätsnahen, aber dennoch abgesicherten Umgebung zu testen.
    Jüngst sind auch viele Hochschulen und Universitäten Ziel von Cyberangriffen geworden, oftmals mit dem Ziel Geld zu erpressen~\parencite{hhnGehackt}.
    Eine häufig zur Kommunikation genutzte Software an Universitäten und Hochschulen ist das Lernmanagementsystem \ac{ilias}, das in der Vergangenheit ebenfalls verschiedene Schwachstellen aufwies, die ein Einfallstor für Hacker darstellten.
    Um eine möglichst realistische Lernumgebung des zu entwickelnden TryHackMe Raumes zu schaffen, greift der zu entwickelnde TryHackMe Raum auf eben diese \ac{ilias} Schwachstellen zurück.


    \subsection{Zielsetzung}
    \label{subsec:zielsetzung}
    Das oberste Ziel des zu TryHackMe Raumes besteht in der Wissensvermittlung bzw. dem Prüfen erworbener Kenntnisse.
    Indem wir interessierte Personen mittels einer Storyline durch einen Raum mit realistischen und in der Vergangenheit aufgetretener Schwachstellen leiten, sollen diese ihr Können unter Beweis stellen.
    Dies soll das Verständnis für die Bedeutung der IT-Sicherheit in der heutigen digitalen Landschaft fördern und gleichzeitig zukünftige Sicherheitsexperten bestmöglich auf Bedrohungen einstellen.
    Die Geschichte handelt von einem jungen Studenten namens TimDerHackerboy, der sich im Laufe seines Kurses \glqq Grundlagen der sicheren Software-Entwicklung\grqq\ zu einer besseren Note hackt.
    Dabei begegnet er unter anderem Schwachstellen wie dem \ac{xss}, ImageTragick oder einer Fehlkonfiguration, die bei geschicktem Ausnutzen eine Privilege Escalation zulässt.
    Es gilt demnach, geeignete Schwachstellen zu bestimmen und diese in die Storyline rund um TimDerHackerboy einzuarbeiten.
    Passende Technologien sind auszuwählen und die \ac{vm}s, zur späteren Einbindung in den TryHackMe Raum, auszuarbeiten.
    Abschließend gilt es, alle Komponenten in einem ansprechend designten TryHackMe Raum zu vereinen.




    \begin{table}[H]
        \caption{Übersicht der SUS Ergebnisse aus dem Screening der Enable Studie}
        \label{tab:sus-ergebnisse-screening}
        \centering
        \begin{threeparttable}
            \begin{tabularx}{\linewidth}{|X|X|}
                \hline
                \textbf{Interference Effect} & \textbf{Value Ranges}  \\
                \hline
                Glare & 2,38 \\
                \hline
                Brightness\tnote{1} & 2,38 \\
                \hline
                Darkness & 0,79\\
                \hline
                4 & 0  \\
                \hline
                5\tnote{1} & 3,97 \\
                \hline
                6\tnote{1} & 7,14 \\
                \hline
                7 & 0  \\
                \hline
                8 & 0  \\
                \hline
                9 & 1,59 \\
                \hline
                10 & 1,59 \\
                \hline
            \end{tabularx}

            \begin{tablenotes}
                \item[1] Durch das Screening der Enable-Fragebogenrückläufer (Abschnitt~\ref{subsec:screening-ergebnisse}) als \textit{auffällig} markiertes Item.
                \item[*] Zur Auswertung werden die Antwortmöglichkeiten des SUS mit Zahlen kodiert. stimme gar nicht zu --> 1; stimme voll zu --> 5.
            \end{tablenotes}
        \end{threeparttable}
    \end{table}





\end{document}

