%Preamble
\documentclass[11pt, a4paper,onecolumn ,titlepage]{article}
%\documentclass[man,floatsintext]{apa7}


%Packages
\usepackage[utf8]{inputenc}
\usepackage{times}
\usepackage[babel=true]{csquotes}
\usepackage[T1]{fontenc}
\usepackage{lmodern}
\usepackage{amsmath,amssymb,amstext}
\usepackage{graphicx}
\usepackage{xstring}
\usepackage{suffix}
\usepackage[nohyperlinks, printonlyused, withpage, smaller]{acronym}
\usepackage{hyperref}
\usepackage{tabularx}
\usepackage{etoolbox}
\usepackage[nooneline]{caption}
\usepackage{float}
\usepackage{pdfpages}
\usepackage{microtype}
\usepackage{booktabs}
\usepackage{abstract}
\usepackage{tablefootnote}
\usepackage{blkarray}
\usepackage{shorttoc}
\usepackage[shortlabels]{enumitem}
\usepackage{threeparttable}
\usepackage{arabicfnt}
\usepackage{tocbibind}

%Einbindung der Referenzen in das Dokument via BiblaTeX
\usepackage[style=apa,backend=biber]{biblatex}
\DeclareLanguageMapping{american}{american-apa}
\addbibresource{main.bib}


%Zum Einfügen von Code
\usepackage{listings}
\usepackage{color}
\definecolor{dkgreen}{rgb}{0,0.6,0}
\definecolor{gray}{rgb}{0.5,0.5,0.5}
\definecolor{mauve}{rgb}{0.58,0,0.82}
\lstset{frame=tb,
    language=python,
    aboveskip=3mm,
    belowskip=3mm,
    showstringspaces=false,
    columns=flexible,
    basicstyle={\small\ttfamily},
    numbers=none,
    numberstyle=\tiny\color{gray},
    keywordstyle=\color{blue},
    commentstyle=\color{dkgreen},
    stringstyle=\color{mauve},
    breaklines=true,
    breakatwhitespace=true,
    tabsize=2
    numbers=left
}


%Document
\begin{document}

%Titelseite
    \begin{titlepage}
        \begin{center}


            \vspace*{1cm}

            {\large \textbf{Quantifying the Robustness of Image Segmentation Algorithms for Medical Assistance Systems}}

            \vspace{1.5cm}

            \begin{figure}[H]
                \centering
                \begin{minipage}[b]{.4\linewidth} % [b] => Ausrichtung an \caption
                    \includegraphics[width=\linewidth]{Title/hhn}
                \end{minipage}\label{fig:unis}
                \hspace{.005\linewidth}% Abstand zwischen Bilder
                \begin{minipage}[b]{.4\linewidth} % [b] => Ausrichtung an \caption
                    \includegraphics[width=\linewidth]{Title/hd}
                \end{minipage}\label{fig:umut}
                \hspace{.005\linewidth}% Abstand zwischen Bilder

            \end{figure}

            \begin{figure}[H]
                \centering
                \begin{minipage}[b]{0.8\linewidth} % [b] => Ausrichtung an \caption
                    \includegraphics[width=\linewidth]{Title/dkfz}
                \end{minipage}\label{fig:dkfz}
            \end{figure}



            \vfill

            {\centering
            Prof. Dr. rer. nat. Alexander Windberger (HHN) \\
            Dr. Annika Reinke (DKFZ)
            }

            \vspace{1.0cm}


            \vspace{0.8cm}

            {\centering
            Lennart Kremp \\
            lkremp@stud.hs-heilbronn.de \\
            Matr.-Nr.: 209437 \\

            \today{}, Heilbronn}


        \end{center}

    \end{titlepage}


    \shorttoc{Table of Contents}{1} %Inhaltübersicht ohne Unterpunkte
    \pagebreak
    \tableofcontents
    \vfill
    \pagebreak


%%%%%%%%%%%%%%%%%%%%%%%%%%%%%%%%%%%%%%%%%%%%%%%%%%%%%%%%%%%%%%%%%%%%%%%%%%%%%%%%%
%	1.Einleitung
%%%%%%%%%%%%%%%%%%%%%%%%%%%%%%%%%%%%%%%%%%%%%%%%%%%%%%%%%%%%%%%%%%%%%%%%%%%%%%%%%
    \fill
    \newpage


    \section{Introduction}
    \label{sec:introduction}

    \subsection{Background and Motivation}
    \label{subsec:background}
    Image Segmentation, a pivotal technique in the medical domain, enables the detection of tissue abnormalities, such
    as tumors, and assists in segmenting medical imaging from X-Rays, CT scans, and MRIs\parencite{kurbig-2005}.
    Another application of image segmentation is to replace current tracking systems, that require additional hardware
    like magnetic field based tracking\parentcite{bianchi-2019} (Bianchi et al., 2019,p.2) or optical trackers (Koivukangas et al., 2013,p.2).
    While these applications are groundbreaking, the success of image segmentation hinges on two critical pillars -
    accuracy and robustness. Accurate segmentation ensures that the features of medical images are correctly identified,
    facilitating precise diagnosis. But in the medical field not only accuracy is important, but also the robustness of
    the predictions. For a physician to make informed decisions aided by an image segmentation algorithm, it needs to
    be reliable in a wide variety of different settings. Optimally, the algorithm rejects all images it can not make a
    reliable prediction on and it is additionally ensured, that the number of rejected images is as small as possible
    (Deserno, 2008). The concept of robustness in image segmentation, will be delved into in greater detail in
    subsequent chapters. To ensure that a given algorithm is fit to be used in the medical field, there is a need for
    an independent certification. Therefore, there has to be an objective way to quantify the robustness of image
    segmentation algorithms. This study seeks to find a transparent measure by developing a comprehensive testing
    framework able to evaluate them in different scenarios. Anyone releasing new image segmentation algorithms for
    the medical field could then use this framework to certify their product. This helps medical practitioners make
    informed decisions with an easy to read robustness scale and improves patient safety by ensuring only adequate
    software is used. Developers could benefit from the transparent robustness scale, because it shows their software
    meets standards for accuracy and robustness, which has long been the norm for medical devices, because it reduces
    the risk of recalls due to defects and potential lawsuits (The Importance of ISO 13485 Certification to Medical
    Device Manufacturing | Smithers, 05.08.2023.) The platform that is being developed quantifies robustness, has an
    open Interface to enable broad usage and can be extended to test the image segmentation with different types of
    interference effects. Because the process is automated, it helps to provide a cheap and easy way to test an image
    segmentation algorithm for its readiness to be deployed in the medical field.




    \begin{table}[H]
        \caption{Übersicht der SUS Ergebnisse aus dem Screening der Enable Studie}
        \label{tab:sus-ergebnisse-screening}
        \centering
        \begin{threeparttable}
            \begin{tabularx}{\linewidth}{|X|X|}
                \hline
                \textbf{Interference Effect} & \textbf{Value Ranges} \\
                \hline
                Glare                        & 2,38                  \\
                \hline
                Brightness\tnote{1}          & 2,38                  \\
                \hline
                Darkness                     & 0,79                  \\
                \hline
                4                            & 0                     \\
                \hline
                5\tnote{1}                   & 3,97                  \\
                \hline
                6\tnote{1}                   & 7,14                  \\
                \hline
                7                            & 0                     \\
                \hline
                8                            & 0                     \\
                \hline
                9                            & 1,59                  \\
                \hline
                10                           & 1,59                  \\
                \hline
            \end{tabularx}

            \begin{tablenotes}
               % \item[1] Durch das Screening der Enable-Fragebogenrückläufer (Abschnitt~\ref{subsec:screening-ergebnisse}) als \textit{auffällig} markiertes Item.
               % \item[1] Durch das Screening der Enable-Fragebogenrückläufer (Abschnitt~\ref{subsec:screening-ergebnisse}) als \textit{auffällig} markiertes Item.
                \item[*] Zur Auswertung werden die Antwortmöglichkeiten des SUS mit Zahlen kodiert. stimme gar nicht zu --> 1; stimme voll zu --> 5.
            \end{tablenotes}
        \end{threeparttable}
    \end{table}

\printbibliography
\end{document}

