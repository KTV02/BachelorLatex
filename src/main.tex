%Preamble
\documentclass[10pt, a4paper,onecolumn ,titlepage]{article}
%\documentclass[man,floatsintext]{apa7}

%Einbindung der Referenzen in das Dokument via BiblaTeX
\usepackage[ngerman]{babel}
\usepackage[style=german,german=quotes]{csquotes}
\usepackage[style=apa,backend=biber]{biblatex}
\DeclareLanguageMapping{american}{american-apa}
\addbibresource{main.bib}

%Packages
\usepackage[utf8]{inputenc}
\usepackage[T1]{fontenc}
\usepackage{lmodern}
\usepackage{amsmath,amssymb,amstext}
\usepackage{graphicx}
\usepackage{xstring}
\usepackage{suffix}
\usepackage[nohyperlinks, printonlyused, withpage, smaller]{acronym}
\usepackage{hyperref}
\usepackage{tabularx}
\usepackage{etoolbox}
\usepackage[nooneline]{caption}
\usepackage{float}
\usepackage{pdfpages}
\usepackage{microtype}
\usepackage{booktabs}
\usepackage{abstract}
\usepackage{tablefootnote}
\usepackage{blkarray}
\usepackage{shorttoc}
\usepackage[shortlabels]{enumitem}
\usepackage{threeparttable}
\usepackage{arabicfnt}
\usepackage{tocbibind}


%Ueberschrieben des kewyowrds befehl
\providecommand{\keywords}[1]
{
    \small
    \textbf{\textit{Keywords --}} #1
}



%Documents
\begin{document}

%Titelseite
    \begin{titlepage}
        \begin{center}

            \vspace*{1cm}

            {\large \textbf{Entwicklung eines TryHackMe Raumes}}

            \vspace{0.5cm}
            zur Sicherheitsschulung von Schwachstellen - aufgezeigt am Lernmanagementsystem ILIAS

            \vspace{1.5cm}

            \textbf{Chris Benz} \\
            \small{chbenz@stud.hs-heilbronn.de}
            \\
            \vspace{0.2cm}
            \textbf{Umut Mehmet Eke}\\
            \small{ueke@stud.hs-heilbronn.de}
            \\
            \vspace{0.2cm}
            \textbf{Lennart Kremp} \\
            \small{lkremp@stud.hs-heilbronn.de}
            \\
            \vspace{0.2cm}
            \textbf{Frederik Spieß}\\
            \small{frspiess@stud.hs-heilbronn.de}


            \vfill

            \begin{figure}[H]
                \centering
                \begin{minipage}[b]{.13\linewidth} % [b] => Ausrichtung an \caption
                    \includegraphics[width=\linewidth]{author_pictures/chris_2}
                \end{minipage}\label{fig:chris}
                \hspace{.005\linewidth}% Abstand zwischen Bilder
                \begin{minipage}[b]{.13\linewidth} % [b] => Ausrichtung an \caption
                    \includegraphics[width=\linewidth]{author_pictures/umut}
                \end{minipage}\label{fig:umut}
                \hspace{.005\linewidth}% Abstand zwischen Bilder
                \begin{minipage}[b]{.13\linewidth} % [b] => Ausrichtung an \caption
                    \includegraphics[width=\linewidth]{author_pictures/chris_2}
                \end{minipage}\label{fig:lennart}
                \hspace{.005\linewidth}% Abstand zwischen Bilder
                \begin{minipage}[b]{.13\linewidth} % [b] => Ausrichtung an \caption
                    \includegraphics[width=\linewidth]{author_pictures/chris_2}
                \end{minipage}\label{fig:frederik}
            \end{figure}


            \vfill

            Durchführung im Rahmen der Veranstaltung \\ \glqq Praktikum sichere Software-Entwicklung\grqq{} an der Hochschule Heilbronn

            \vspace{0.2cm}

            Beaufsichtigt durch Prof. Dr.-Ing. Andreas Mayer

            \vspace{1.0cm}


            \vspace{0.8cm}

            Fakultät für Informatik\\
            Hochschule Heilbronn\\
            \today{}, Heilbronn

        \end{center}

    \end{titlepage}

%%%%%%%%%%%%%%%%%%%%%%%%%%%%%%%%%%%%%%%%%%%%%%%%%%%%%%%%%%%%%%%%%%%%%%%%%%%%%%%%%
%	Abstract
%%%%%%%%%%%%%%%%%%%%%%%%%%%%%%%%%%%%%%%%%%%%%%%%%%%%%%%%%%%%%%%%%%%%%%%%%%%%%%%%%

    \renewcommand*\abstractname{\flushleft\textbf{Abstract}\hfill}
    \begin{abstract}
        \hline
        \vspace{0.5cm}
        \noindent
        \textbf{Hintergrund}:  In einer immer stärker miteinander vernetzten Welt wird eines immer wichtiger, die Sicherheit und der Schutz digitaler Infrastruktur.
        Dies zeigt auch eine immer weiter zunehmenden Anzahl an Cyberangriffen, die es nötig macht, seine IT-Infrastruktur immer im Blick zu behalten und bestmöglich vor ungewollten Angriffen zu schützen.
        Geschultes IT-Personal mit fundierten Kenntnissen im Umgang mit Schwachstelle ist somit unerlässlich.

        \vspace{0.5cm}
        \noindent
        \textbf{Zielsetzung}: Das Ziel dieses Projekts ist das Erstellen eines TryHackMe Raumes, der Anhand einer ansprechenden Storyline
        Wissen im Bereich IT-Sicherheit vermittelt und vorhandene Kenntnisse auf die Probe stellt.

        \vspace{0.5cm}
        \noindent
        \textbf{Methoden}: An erster Stelle steht das Bestimmen geeigneter Schwachstellen in ILIAS und die Anpassung an die Storyline.
        Daraus resultierend gilt es die passenden Technologien auszuwählen und die \ac{vm} zur späteren Einbindung in den TryHackMe Raum auszuarbeiten.
        Abschließen gilt es alle Komponenten in einem ansprechend designten TryHackMe Raum zu vereinen.


        \vspace{0.5cm}
        \noindent
        \textbf{Ergebnisse}: Insgesamt wurden zwei \ac{vm}s mit den Ilias Versionen 5.2.3 und 5.2.4 erstellt.
        Eine Storyline rund um den Studenten TimDerHackerBoy führt die Nutzer des Raumes durch die beiden Maschinen.
        Um den Raum erfolgreich abzuschließen sind unter anderem verschiedene Schwachstellen auszunutzen, drei Flags zu finden und den erfolgreichen Umgang mit verschiedenen Tools wie Hydra unter Beweis zustellen.

        \vspace{0.5cm}
        \hline
        \vspace{3cm}
        \noindent
        \keywords{Schwachstellen, Ilias, ImageMagick, Privilage Escalation in Linux, XSS}
    \end{abstract}

%%%%%%%%%%%%%%%%%%%%%%%%%%%%%%%%%%%%%%%%%%%%%%%%%%%%%%%%%%%%%%%%%%%%%%%%%%%%%%%%%
%	1.Inhaltsverzeichnis
%%%%%%%%%%%%%%%%%%%%%%%%%%%%%%%%%%%%%%%%%%%%%%%%%%%%%%%%%%%%%%%%%%%%%%%%%%%%%%%%%


    \shorttoc{Inhaltsübersicht}{1}
    \pagebreak
    \tableofcontents
    \vfill
    \pagebreak


%%%%%%%%%%%%%%%%%%%%%%%%%%%%%%%%%%%%%%%%%%%%%%%%%%%%%%%%%%%%%%%%%%%%%%%%%%%%%%%%%
%	1.Einleitung
%%%%%%%%%%%%%%%%%%%%%%%%%%%%%%%%%%%%%%%%%%%%%%%%%%%%%%%%%%%%%%%%%%%%%%%%%%%%%%%%%
    \fill
    \newpage
    \section{Einleitung}
    \label{sec:einleitung}

    \subsection{Gegenstand und Motivation}
    \label{subsec:gegenstand-motivation}
    \footnote{In dieser Arbeit wird aus Gründen der besseren Lesbarkeit das generische Maskulinum verwendet. Weibliche und anderweitige Geschlechteridentitäten werden dabei ausdrücklich mitgemeint, soweit es für die Aussage erforderlich ist.}
    In einer immer stärker miteinander vernetzten Welt wird eines immer wichtiger, die Sicherheit und der Schutz digitaler Infrastruktur.
    Dies zeigt auch eine immer weiter zunehmenden Anzahl an Cyberangriffen, die es nötig macht, seine IT-Infrastruktur immer im Blick zu behalten und bestmöglich vor ungewollten Angriffen zu schützen.
    Geschultes IT-Personal mit fundierten Kenntnissen in Bereichen wie.
    Eine Möglichkeit ein solches Wissen zu erlangen, bietet die Lernplattform \href{https://tryhackme.com/}{TryHackMe}.
    Sie bietet interessierten Personen die Möglichkeit sich in Lern- und Challengeräumen umfangreiches Wissen im Bereich Ethical Hacking anzueignen.
    Und das erworbene Wissen in einer realitätsnahen, aber dennoch abgesicherten Umgebung zu testen.
    Jüngst sind auch viele Hochschulen und Universitäten Ziel von Cyberangriffen geworden, oftmals mit dem Ziel Geld zu erpressen.
    Eine häufig zur Kommunikation genutzte Software ist das Lernmanagementsystem ILIAS, das in der Vergangenheit ebenfalls verschiedene Schwachstellen aufwies, die ein Einfallstor für Hacker darstellen.
    Um eine möglichst realistische Lernumgebung des zu entwickelnden TryHackMe Raumes zu schaffen,


    \subsection{Zielsetzung}
    \label{subsec:zielsetzung}
    Das oberste Ziel des zu entwickelnden TryHackMe Raums besteht in der Wissensvermittlung bzw. dem Prüfen erworbener Kenntnisse.
    Indem wir interessierte Personen mittels einer Storyline durch einen mit realistischen und in der Vergangenheit aufgetretener Schwachstellen leiten, sollen diese ihr Können unter Beweis stellen.
    Dies soll das Verständnis für die Bedeutung der IT-Sicherheit in der heutigen digitalen Landschaft fördern und gleichzeitig zukünftige Sicherheitsexperten bestmöglich auf Bedrohungen einstellen.
    Die Geschichte handelt von einem jungen Studenten namens TimDerHackerboy, der sich im Laufe seines Kurses \glqq Grundlagen der sicheren Software-Entwicklung\grqq\ zu einer besseren Note hackt.
    Dabei begegnet er unter anderem Schwachstellen wie dem \ac{xss}, ImageTragick oder einer Misskonfiguration, die bei geschicktem Ausnutzen eine Privilege Escalation zulässt.




%%%%%%%%%%%%%%%%%%%%%%%%%%%%%%%%%%%%%%%%%%%%%%%%%%%%%%%%%%%%%%%%%%%%%%%%%%%%%%%%%
%	2.Grundlagen
%%%%%%%%%%%%%%%%%%%%%%%%%%%%%%%%%%%%%%%%%%%%%%%%%%%%%%%%%%%%%%%%%%%%%%%%%%%%%%%%%
    \fill
    \newpage
    \section{Grundlagen}
    \label{sec:grundlagen}

    \subsection{ILIAS}
    \label{subsec:ilias}
    Die Abkürzung ILIAS beschreibt ein \ac{ilias} .
    Bei der Software handelt es sich nicht um eine Lernplattform, sondern um ein System mit dem sich eine solche betreiben lässt.
    Die Software fällt unter die Kategorie Open-Source und ist somit für jedermann zugänglich und wird seit dem Jahre 2000, aufgrund großen Interesses, unter der GNU General Public Licence veröffentlicht.
    Mittlerweile erscheint \ac{ilias} in der achten Version und ermöglicht Lehrenden und Lernenden eine möglichst unkomplizierte Kommunikation.
    \ac{ilias} ist wie eine Bibliothek zu verstehen, die es ermöglicht Wissen in Kursen möglichst einfach bereitzustellen und zu managen.
    Mittlerweile wurde diese Kernkompetenz zudem durch Module zur Kommunikation oder auch dem Durchführen von Onlineprüfungen ergänzt.
    Nutzer sind mittlerweile nicht nur Hochschulen und Universitäten, sondern auch Akademien, die NATO und die Bayerische Polizei.

    \subsection{Cross-Site-Scripting}
    \label{subsec:CrossSiteScripting}
    Bei \ac{xss} handelt es sich um eine weit verbreitete Schwachstelle, die zudem von der \ac{owasp} geführt wird.
    XSS-Angriffe sind eine Art von Injektion, bei der bösartige Skripte in ansonsten gutartige und vertrauenswürdige Websites eingeschleust werden.
    XSS-Angriffe treten auf, wenn ein Angreifer eine Webanwendung nutzt, um bösartigen Code, in der Regel in Form eines browserseitigen Skripts, an einen anderen Endbenutzer zu senden~\parencite{xss}.
    Eine Ursache für das Auftreten von XSS-Schwachstellen ist das Vertrauen in den Nutzer oder schlichtweg eine nicht vorhandene Validierung der Nutzereingaben.
    Angreifer mit unethischen Absichten erhalten so einen Weg, der es ihnen erlaubt, schädliche Skripte auszuführen und so vertrauliche Informationen zu erlangen.
    Je nach Schwachstelle kann es sich dabei beispielsweise um Session-Cookies oder Zugangsdaten handeln.
    Zudem findet eine Einteilung in zwei Arten von XSS-Angriffen statt:
    \\
    (1) \textbf{Reflected \ac{xss} Attacken} beschreiben eine Attacke, bei der das \ac{xss} Skript vom Webserver reflektiert wird.
    Dies kann unter anderem in Form einer Fehlermeldung oder Suchergebnissen der Fall sein.
    \\
    (2) \textbf{Stored \ac{xss} Attacken} hingegen beschreiben eine Attacke, bei der das \ac{xss} Skript eingeschleust und anschließend auf dem verwundbaren Zielserver gespeichert wird.
    Dies ermöglicht ein Ausführen des schädliche \ac{xss} Skripts, nach jedem Aufruf des Zielservers durch unbedarfte Nutzer.
    Beliebte Beispiele zum Platzieren eines solchen \ac{xss} Skripts sind Forenbeiträge, Kommentare oder Datenbankeinträge.


    \subsubsection{CVE-2018-5688}
    \label{subsubsec:CVE-2018-5688}
    Die Schwachstelle \href{https://www.cve.org/CVERecord?id=CVE-2018-5688}{CVE-2018-5688} tritt in ILIAS Versionen vor 5.2.4 auf.
    Durch Ausnutzung dieser Schwachstelle kann schädlicher \ac{xss} Skriptcode eingeschleust und ausgeführt werden, was potenziell zu Sicherheitsproblemen führen kann.
    Diese Schwachstelle betrifft den cmd-Parameter der displayHeader-Funktion womit .php Dateien in der Setup-Komponente der Ilias-Software ausgelesen werden können~\parencite{xssExploitDb}.
    Zum Ausnutzen der Schwachstelle muss ein Angreifer nicht eingeloggt sein.
    Mit einer CVSS Bewertung von 4.3 ist die Schwachstelle nicht als extrem gefährlich eingestuft, was daran liegt, dass die möglichen Auswirkungen auf das System moderat sind~\parencite{xssCVEDetails}.

    \subsection{ImageMagick}
    \label{subsec:ImageMagick}
    \href{https://imagemagick.org}{ImageMagick} ist eine weitverbreitete Open-Source Software zur Anzeige, Konvertierung und Bearbeitung von Bilddateien.
    Es wird ein command-line Interface, sowie APIs für die Integration in eigene Softwareprojekte bereitgestellt.
    Die Software bietet dabei einen großen Umfang an Funktionen und unterstützt eine Vielzahl an gängigen Dateiformaten, weshalb sie eine große Nutzerschaft in Bereichen wie der Webentwicklung, wissenschaftlichen Forschung, medizinischen Bildgebung und vielen weiteren aufweisen kann.
    Eine der wichtigsten Funktionen, die ImageMagick zu bieten hat, ist das Verarbeiten von Skripten und die Möglichkeit zur Automatisierung.
    Immer gleich Abläufe für eine Großzahl an Bildern können so vereinfacht werde, weshalb die Software auch in Ilias eingesetzt wird~\parencite{imagemagick}.


    \subsubsection{CVE-2016-3718 (ImageTragick)}
    \label{subsubsec:CVE-2016-3718}
    Die Schwachstelle \href{https://www.cve.org/CVERecord?id=CVE-2016-3718}{CVE-2016-3718}, auch \glqq ImageTragick\grqq genannt, tritt in den Versionen 6.9.3-10 und 7.x vor 7.0.1-1 von ImageMagick auf.
    Die Sicherheitslücke ist im Jahre 2016 erstmal entdeckt worden und betrifft die Verarbeitung von Bilddateien mit der ImageMagick Software-Bibliothek.
    Durch \glqq ImageTragick\grqq ist es möglich, Schadcode in Form einer Bilddatei in ein System einzuschleusen und anschließend serverseitig auszuführen.
    Da es sich bei ImageMagick um eine Software-Bibliothek handelt, die wie in~\ref{subsec:ImageMagick} bereits erwähnt eine große Nutzerzahl aufweist, war die Schwachstelle zum Zeitpunkt ihrer Entdeckung durch die Sicherheitsfirma Check Point in einer Vielzahl von Softwareprodukten zu finden~\parencite{imageTragicReport}.
    ILIAS verwendet ImageMagic zur Verarbeitung der Profilbilder ihrer Nutzer, weshalb auch ILIAS je nach verwendeter ImageMagick Version von der Sicherheitslücke betroffen war.

    \subsection{Privilege Escalation}
    \label{subsec:PrivilegeEscalation}

    \subsubsection{CVE-2019-14287 (Sudo)}
    \label{subsubsec:sudo}

    \subsubsection{Misskonfiguration (/etc/pssswd)}
    \label{subsubsec:misskonfiguration}







%%%%%%%%%%%%%%%%%%%%%%%%%%%%%%%%%%%%%%%%%%%%%%%%%%%%%%%%%%%%%%%%%%%%%%%%%%%%%%%%%
%	3. Technologien / Methoden und Werkzeuge
%%%%%%%%%%%%%%%%%%%%%%%%%%%%%%%%%%%%%%%%%%%%%%%%%%%%%%%%%%%%%%%%%%%%%%%%%%%%%%%%%
    \fill
    \newpage

    \section{Technologien}
    \label{sec:technologien}

    \subsection{Docker}
    \label{subsec:docker}
    Docker ist eine Open-Source-Plattform, die es Entwicklern ermöglicht, Anwendungen in Containern zu erstellen, zu verpacken und auszuführen.\n 
    Container sind eigenständige und isolierte Einheiten, die alle erforderlichen Abhängigkeiten enthalten, um Anwendungen auf verschiedenen Systemen konsistent auszuführen.\autocite{docker}
    \n Theoretisch stellt der User \glqq sturai\grqq\ ein ILIAS-Image über Docker Hub bereit (\href{https://hub.docker.com/r/sturai/ilias#!}{s. hier}).
    Jedoch konnten wir dieses nicht benutzen, da die angebotenen Versionen unsere Exploits nicht mehr unterstützen bzw. diese darin gefixt sind.\n
    Unsere Lösung
    \subsection{VM mit einem Ubuntu Betriebssystem}
    \label{subsec:ubuntu}

    \subsection{MySQL Datenbank}
    \label{subsec:mysqlDatenbank}

    \subsection{Apache Server mit einem Debian Betriebssystem}
    \label{subsec:apacherServer}

    \subsection{ILIAS mit ImageMagick}
    \label{subsec:iliasTechnologie}


%%%%%%%%%%%%%%%%%%%%%%%%%%%%%%%%%%%%%%%%%%%%%%%%%%%%%%%%%%%%%%%%%%%%%%%%%%%%%%%%%
%	4. Storyline und Musterlösung
%%%%%%%%%%%%%%%%%%%%%%%%%%%%%%%%%%%%%%%%%%%%%%%%%%%%%%%%%%%%%%%%%%%%%%%%%%%%%%%%%
    \fill
    \newpage

    \section{Storyline}
    \label{sec:storyline}

    (\href{https://tryhackme.com/room/professorstudent}{Zum TryHackMe Raum})

    %Bilder vom TryHackMe Raum einfügen

    \subsection{Virtuelle Maschine 1}
    \label{subsec:vm1}

    \subsection{Virtuelle Maschine 2}
    \label{subsec:vm2}




%%%%%%%%%%%%%%%%%%%%%%%%%%%%%%%%%%%%%%%%%%%%%%%%%%%%%%%%%%%%%%%%%%%%%%%%%%%%%%%%%
%	4.Lessons Learned
%%%%%%%%%%%%%%%%%%%%%%%%%%%%%%%%%%%%%%%%%%%%%%%%%%%%%%%%%%%%%%%%%%%%%%%%%%%%%%%%%
    \fill
    \newpage
    \section{Lessons Learned}
    \label{sec:lessonsLearned}

    \subsection{Analyseergebnisse der Enable-Fragebogenrückläufer}
    \label{subsec:screening-ergebnisse}

    \subsubsection{Ergebnisse SUS}
    \label{subsubsec:screening-ergebnisse-sus}



%%%%%%%%%%%%%%%%%%%%%%%%%%%%%%%%%%%%%%%%%%%%%%%%%%%%%%%%%%%%%%%%%%%%%%%%%%%%%%%%%
%	7.Danksagung
%%%%%%%%%%%%%%%%%%%%%%%%%%%%%%%%%%%%%%%%%%%%%%%%%%%%%%%%%%%%%%%%%%%%%%%%%%%%%%%%%
    \vspace{5cm}
    \hline
    \vspace{1cm}
    \noindent
    \textbf{Danksagung}
    \\
    \\
    Wir bedanken uns bei \textit{Prof. Dr.-Ing. Andreas Mayer} für die Betreuung im Rahmen des Kurses Praktikum sichere Software-Entwicklung an der Hochschule Heilbronn.
    \vspace{1cm}
    \hline
    \vspace{2cm}

%%%%%%%%%%%%%%%%%%%%%%%%%%%%%%%%%%%%%%%%%%%%%%%%%%%%%%%%%%%%%%%%%%%%%%%%%%%%%%%%%
%	7.Bibliographie
%%%%%%%%%%%%%%%%%%%%%%%%%%%%%%%%%%%%%%%%%%%%%%%%%%%%%%%%%%%%%%%%%%%%%%%%%%%%%%%%%
    \fill
    \newpage
    \section{Literaturverzeichnis}
    \label{sec:bibliographie}
    \printbibliography[title=""]

%%%%%%%%%%%%%%%%%%%%%%%%%%%%%%%%%%%%%%%%%%%%%%%%%%%%%%%%%%%%%%%%%%%%%%%%%%%%%%%%%
%	8.Abkürzungsverzeichnis
%%%%%%%%%%%%%%%%%%%%%%%%%%%%%%%%%%%%%%%%%%%%%%%%%%%%%%%%%%%%%%%%%%%%%%%%%%%%%%%%%
    \fill
    \newpage

    \section{Abkürzungsverzeichnis}
    \label{sec:abkuerzungsverzeichnis}
    \begin{acronym}
        \acro{xss}[XSS]{Cross-Site Scripting}
        \acro{ilias}[ILIAS]{Integriertes Lern-, Informations- und Arbeitskooperations-System}
        \acro{owasp}[OWASP]{Open Web Application Security Project}
        \acro{vm}[VM]{Virtuellen Maschinen}
    \end{acronym}


%%%%%%%%%%%%%%%%%%%%%%%%%%%%%%%%%%%%%%%%%%%%%%%%%%%%%%%%%%%%%%%%%%%%%%%%%%%%%%%%%
%	9.Anhang
%%%%%%%%%%%%%%%%%%%%%%%%%%%%%%%%%%%%%%%%%%%%%%%%%%%%%%%%%%%%%%%%%%%%%%%%%%%%%%%%%
    \fill
    \newpage
    \section{Anhang}
    \label{sec:Anhang}




\end{document}

