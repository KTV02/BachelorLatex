%Preamble
\documentclass[10pt, a4paper,onecolumn ,titlepage]{article}
%\documentclass[man,floatsintext]{apa7}

%Einbindung der Referenzen in das Dokument via BiblaTeX
\usepackage[ngerman]{babel}
\usepackage[style=german,german=quotes]{csquotes}
\usepackage[style=apa,backend=biber]{biblatex}
\DeclareLanguageMapping{american}{american-apa}
\addbibresource{main.bib}

%Packages
\usepackage[utf8]{inputenc}
\usepackage[T1]{fontenc}
\usepackage{amsmath,amssymb,amstext}
\usepackage{graphicx}
\usepackage{xstring}
\usepackage{suffix}
\usepackage[nohyperlinks, printonlyused, withpage, smaller]{acronym}
\usepackage{hyperref}
\usepackage{tabularx}
\usepackage{etoolbox}
\usepackage[nooneline]{caption}
\usepackage{float}
\usepackage{pdfpages}
\usepackage{microtype}
\usepackage{booktabs}
\usepackage{abstract}
\usepackage{tablefootnote}
\usepackage{blkarray}
\usepackage{shorttoc}
\usepackage[shortlabels]{enumitem}
\usepackage{threeparttable}

%Ueberschrieben des kewyowrds befehl
\providecommand{\keywords}[1]
{
    \small
    \textbf{\textit{Keywords --}} #1
}



%Documents
\begin{document}

%Titelseite
    \begin{titlepage}
        \begin{center}

            \vspace*{1cm}

            {\large \textbf{Entwickeln eines TryHackMe Raums}}

            \vspace{0.5cm}
            zur Sicherheitsschulung von Schachstellen, aufgezeigt am Lernmanagementsystem ILIAS

            \vspace{1.5cm}

            \textbf{Chris Benz} \\
            \small{chbenz@stud.hs-heilbronn.de}
            \\
            \vspace{0.2cm}
            \textbf{Lennart Kremp}\\
            \small{jkaercher@stud.hs-heilbronn.de}
            \\
            \vspace{0.2cm}
            \textbf{Umut Eke} \\
            \small{sthuerwae@stud.hs-heilbronn.de}
            \\
            \vspace{0.2cm}
            \textbf{Frederik Spieß}\\
            \small{ermeyer@stud.hs-heilbronn.de}


            \vfill

            \begin{figure}[H]
                \centering
                \begin{minipage}[b]{.13\linewidth} % [b] => Ausrichtung an \caption
                    \includegraphics[width=\linewidth]{author_pictures/chris_2}
                \end{minipage}\label{fig:chris}
                \hspace{.005\linewidth}% Abstand zwischen Bilder
                \begin{minipage}[b]{.13\linewidth} % [b] => Ausrichtung an \caption
                    \includegraphics[width=\linewidth]{author_pictures/chris_2}
                \end{minipage}\label{fig:lennart}
                \hspace{.005\linewidth}% Abstand zwischen Bilder
                \begin{minipage}[b]{.13\linewidth} % [b] => Ausrichtung an \caption
                    \includegraphics[width=\linewidth]{author_pictures/chris_2}
                \end{minipage}\label{fig:umut}
                \hspace{.005\linewidth}% Abstand zwischen Bilder
                \begin{minipage}[b]{.13\linewidth} % [b] => Ausrichtung an \caption
                    \includegraphics[width=\linewidth]{author_pictures/chris_2}
                \end{minipage}\label{fig:frederik}
            \end{figure}


            \vfill

            Durchführung im Rahmen der Veranstaltung \\ \glqq Praktikum sichere Software-Entwicklung\grqq{} an der Hochschule Heilbronn

            \vspace{0.2cm}

            Beaufsichtigt durch Prof. Dr.-Ing. Andreas Mayer

            \vspace{1.0cm}


            \vspace{0.8cm}

            Fakultät für Informatik\\
            Hochschule Heilbronn\\
            \today{}, Heilbronn

        \end{center}

    \end{titlepage}

%%%%%%%%%%%%%%%%%%%%%%%%%%%%%%%%%%%%%%%%%%%%%%%%%%%%%%%%%%%%%%%%%%%%%%%%%%%%%%%%%
%	Abstract
%%%%%%%%%%%%%%%%%%%%%%%%%%%%%%%%%%%%%%%%%%%%%%%%%%%%%%%%%%%%%%%%%%%%%%%%%%%%%%%%%

    \renewcommand*\abstractname{\flushleft\textbf{Abstract}\hfill}
    \begin{abstract}
        \hline
        \vspace{0.5cm}
        \noindent
        \textbf{Hintergrund}: Die fortschreitende Digitalisierung ist in allen Bereichen des Lebens angekommen.
        Apps im Allgemeinen, aber auch neue Entwicklungsfelder wie mHealth-Anwendungen erfahren einen starken Anstieg ihrer Nutzerzahlen.
        Dabei ist eine hohe Usability einer der essenziellen Faktoren erfolgreicher Apps.
        Eine fundierte und frühzeitige Beurteilung der Usability einer App bietet den Entwickler*innen die Möglichkeit, Schwachstellen schnell zu beseitigen.
        Es existiert eine Vielzahl verschiedener Fragebögen zur Evaluierung der Usability.
        Oft verwendete Fragebögen sind der System Usability Scale (SUS) sowie der speziell für mHealth-Apps entwickelte mHealth App Usability Questionnaire (MAUQ).
        Neue Daten der am Universitätsklinikum durchgeführten Enable Studie deuten auf Schwierigkeiten beim Verständnis der Items von SUS und MAUQ hin.

        \vspace{0.5cm}
        \noindent
        \textbf{Zielsetzung}: Das Ziel dieses Projekts ist es, festzustellen, ob es bei den Items der Fragebögen SUS und MAUQ Verständnisprobleme gibt.
        Bestätigt sich diese Hypothese, sind auffällige Items zu identifizieren und abschließend jeweils eine überarbeitete Version der Fragebögen zu entwickeln.

        \vspace{0.5cm}
        \noindent
        \textbf{Methoden}: Die in der Enable Studie erhobenen Fragebögen werden in einem ersten Schritt auf Auffälligkeiten und Anmerkungen untersucht,
        um im Anschluss die Items, durch Berechnung der Item Conspicuity Rate (ICR), zu evaluieren.
        In einem zweiten Schritt werden die beiden Fragebögen durch Interviews mit weiteren Proband*innen untersucht, um Informationen über die Ursachen der Verständnisprobleme zu erhalten.
        Hierbei liegt der Fokus auf den, durch die \ac{icr}, als auffällig identifizierten Items.
        Abschließend werden neue Versionen des SUS und MAUQ entworfen.


        \vspace{0.5cm}
        \noindent
        \textbf{Ergebnisse}: Insgesamt wurden 235 MAUQ Fragebögen und 126 SUS Fragebögen einem Screening unterzogen.
        Im Anschluss wurden 37 Proband*innen für eine weitere Erhebung konsultiert.
        Einige Items der beiden Fragebögen sind aufgrund einer Vielzahl an Auffälligkeiten sowie eindeutiger Bemerkungen der Proband*innen, als auffällig einzustufen. %vielleicht auffällig
        Abschließend wurden auf Grundlage der neu gewonnen Erkenntnisse neue Versionen für SUS und MAUQ entwickelt.


        \vspace{0.5cm}
        \hline
        \vspace{3cm}
        \noindent
        \keywords{Schwachstellen, Ilias, ImageMagic, Privilage Escalation in Linux, XSS}
    \end{abstract}

%%%%%%%%%%%%%%%%%%%%%%%%%%%%%%%%%%%%%%%%%%%%%%%%%%%%%%%%%%%%%%%%%%%%%%%%%%%%%%%%%
%	1.Inhaltsverzeichnis
%%%%%%%%%%%%%%%%%%%%%%%%%%%%%%%%%%%%%%%%%%%%%%%%%%%%%%%%%%%%%%%%%%%%%%%%%%%%%%%%%


    \shorttoc{Inhaltsübersicht}{1}
    \pagebreak
    \tableofcontents
    \vfill
    \pagebreak


%%%%%%%%%%%%%%%%%%%%%%%%%%%%%%%%%%%%%%%%%%%%%%%%%%%%%%%%%%%%%%%%%%%%%%%%%%%%%%%%%
%	1.Einleitung
%%%%%%%%%%%%%%%%%%%%%%%%%%%%%%%%%%%%%%%%%%%%%%%%%%%%%%%%%%%%%%%%%%%%%%%%%%%%%%%%%
    \fill
    \newpage
    \section{Einleitung}
    \label{sec:einleitung}

    \subsection{Gegenstand und Motivation}
    \label{subsec:gegenstand-motivation}
    \footnote{In dieser Arbeit wird aus Gründen der besseren Lesbarkeit das generische Maskulinum verwendet. Weibliche und anderweitige Geschlechteridentitäten werden dabei ausdrücklich mitgemeint, soweit es für die Aussage erforderlich ist.}
    In einer immer stärker miteinander vernetzten Welt wird eines immer wichtiger, die Sicherheit und der Schutz digitaler Infrastruktur.
    Dies zeigt auch eine immer weiter zunehmenden Anzahl an Cyberangriffen, die es nötig macht, seine IT-Infrastruktur immer im Blick zu behalten und bestmöglich vor ungewollten Angriffen zu schützen.
    Geschultes IT-Personal mit fundierten Kenntnissen in Bereichen wie.
    Eine Möglichkeit ein solches Wissen zu erlangen, bietet die Lernplattform TryHackMe.
    Sie bietet interessierten Personen die Möglichkeit sich in Lern- und Challengeräumen umfangreiches Wissen im Bereich Ethical Hacking anzueignen.
    Und das erworbene Wissen in einer realitätsnahen, aber dennoch abgesicherten Umgebung zu testen.
    Jüngst sind auch viele Hochschulen und Universitäten Ziel von Cyberangriffen geworden, oftmals mit dem Ziel Geld zu erpressen.
    Eine häufig zur Kommunikation genutzte Software ist das Lernmanagementsystem ILIAS, das in der Vergangenheit ebenfalls verschiedene Schwachstellen aufwies, die ein Einfallstor für Hacker darstellen.
    Um eine möglichst realistische Lernumgebung des zu entwickelnden TryHackMe Raumes zu schaffen,


    \subsection{Zielsetzung}
    \label{subsec:zielsetzung}
    Das oberste Ziel des zu entwickelnden TryHackMe Raums besteht in der Wissensvermittlung bzw. dem Prüfen erworbener Kenntnisse.
    Indem wir interessierte Personen mittels einer Storyline durch einen mit realistischen und in der Vergangenheit aufgetretener Schwachstellen leiten, sollen diese ihr Können unter Beweis stellen.
    Dies soll das Verständnis für die Bedeutung der IT-Sicherheit in der heutigen digitalen Landschaft fördern und gleichzeitig zukünftige Sicherheitsexperten bestmöglich auf Bedrohungen einstellen.
    Die Geschichte handelt von einem jungen Studenten namens TimDerHackerboy der sich im Laufe seines Kurses \glqq Grundlagen der sicheren Software-Entwicklung\grqq zu einer besseren Note hackt.
    Dabei begegnet er unter anderem Schachstellen wie dem \ac{xss} , ImageTragic oder einer Misskonfiguration die bei geschicktem Ausnutzen eine Privilege Escalation zulässt.




%%%%%%%%%%%%%%%%%%%%%%%%%%%%%%%%%%%%%%%%%%%%%%%%%%%%%%%%%%%%%%%%%%%%%%%%%%%%%%%%%
%	2.Grundlagen
%%%%%%%%%%%%%%%%%%%%%%%%%%%%%%%%%%%%%%%%%%%%%%%%%%%%%%%%%%%%%%%%%%%%%%%%%%%%%%%%%
    \fill
    \newpage
    \section{Grundlagen}
    \label{sec:grundlagen}

    \subsection{ILIAS}
    \label{subsec:ilias}
    Die Abkürzung ILIAS beschreibt ein \ac{ilias} .
    Bei der Software handelt es sich nicht um eine Lernplattform, sondern um ein System mit dem sich eine solche betreiben lässt.
    Die Software fällt unter die Kategorie Open-Source und ist somit für jedermann zugänglich und wird seit dem Jahre 2000, aufgrund großen Interesses, unter der GNU General Public Licence veröffentlicht.
    Mittlerweile erscheint \ac{ilias} in der achten Version und ermöglicht Lehrenden und Lernenden eine möglichst unkomplizierte Kommunikation.
    \ac{ilias} ist wie eine Bibliothek zu verstehen, die es ermöglicht Wissen in Kursen möglichst einfach bereitzustellen und zu managen.
    Mittlerweile wurde diese Kernkompetenz zudem durch Module zur Kommunikation oder auch dem Durchführen von Onlineprüfungen ergänzt.
    Nutzer sind mittlerweile nicht nur Hochschulen und Universitäten, sondern auch Akademien, die Nato und die Bayrische Polizei.

    \subsection{Cross-Site-Scripting}
    \label{subsec:CrossSiteScripting}
    Bei \ac{xss} handelt es sich um eine weit verbreitet Schachstelle die zudem von der \ac{owasp} geührt wird.
    XSS-Angriffe sind eine Art von Injektion, bei der bösartige Skripte in ansonsten gutartige und vertrauenswürdige Websites eingeschleust werden.
    XSS-Angriffe treten auf, wenn ein Angreifer eine Webanwendung nutzt, um bösartigen Code, in der Regel in Form eines browserseitigen Skripts, an einen anderen Endbenutzer zu senden. ~\textcite{RN1}
    Ein Ursache für das Auftreten von XSS-Schwachstellen, ist das Vertrauen in den Nutzer oder schlichtweg eine nicht vorhandene Validierung der Nutzereingaben.
    Angreifer mit unethischen Absichten erhalten so einen Weg, der es ihnen erlaubt, schädliche Scripte auszuführen und so vertrauliche Informationen zu erlangen.
    Je nach Schwachstelle kann es sich dabei zum Beispiel um Session-Cookies oder Zugangsdaten handeln.
    Zudem findet eine einteilung in zwei Arten von XSS-Angriffen statt. Ersten,

    \subsubsection{CVE-2018-5688}
    \label{subsubsec:CVE-2018-5688}
    Die Schwachstelle CVE-2018-5688 tritt in ILIAS Versionen vor 5.2.4 auf.

    \subsection{ImageMagic}
    \label{subsec:ImageMagic}

    \subsubsection{CVE-2016-3718 (ImageTragic)}
    \label{subsubsec:CVE-2016-3718}

    \subsection{Privilege Escalation}
    \label{subsec:PrivilegeEscalation}

    \subsubsection{CVE-2019-14287 (Sudo)}
    \laber{subsubsec:sudo}

    \subsubsection{Misskonfiguration (/etc/pssswd)}
    \laber{subsubsec:misskonfiguration}







%%%%%%%%%%%%%%%%%%%%%%%%%%%%%%%%%%%%%%%%%%%%%%%%%%%%%%%%%%%%%%%%%%%%%%%%%%%%%%%%%
%	3. Technologien / Methoden und Werkzeuge
%%%%%%%%%%%%%%%%%%%%%%%%%%%%%%%%%%%%%%%%%%%%%%%%%%%%%%%%%%%%%%%%%%%%%%%%%%%%%%%%%
    \fill
    \newpage

    \section{Technologien}
    \label{sec:technologien}

    \subsection{Prozess Übersicht}
    \label{subsec:process-overview}
    Um jeweils einen neuen Entwurf für die Fragebögen SUS und MAUQ zu erhalten, werden zunächst die während der Enable Studie am Universitätsklinikum Heidelberg
    erhobenen Fragebogenrückläufer ausgewertet.
    Die erhobenen Daten bilden die Grundlage für eine Auswahl \textit{auffälliger Items}.
    In einem anschließenden Interview werden weitere Daten mit Fokus auf die, als auffällig klassifizierten, Items erhoben,
    um im Anschluss jeweils einen sprachlich angepassten Entwurf für SUS und MAUQ auszuarbeiten.
    Abbildung ~\ref{fig:flow-chart-übersicht} visualisiert die gewählte Vorgehensweise und stellt einen kompakten Überblick dar.



%%%%%%%%%%%%%%%%%%%%%%%%%%%%%%%%%%%%%%%%%%%%%%%%%%%%%%%%%%%%%%%%%%%%%%%%%%%%%%%%%
%	4. Storyline und Musterlösung
%%%%%%%%%%%%%%%%%%%%%%%%%%%%%%%%%%%%%%%%%%%%%%%%%%%%%%%%%%%%%%%%%%%%%%%%%%%%%%%%%
    \fill
    \newpage

    \section{Storyline}
    \label{sec:storyline}






%%%%%%%%%%%%%%%%%%%%%%%%%%%%%%%%%%%%%%%%%%%%%%%%%%%%%%%%%%%%%%%%%%%%%%%%%%%%%%%%%
%	4.Lessons Learned
%%%%%%%%%%%%%%%%%%%%%%%%%%%%%%%%%%%%%%%%%%%%%%%%%%%%%%%%%%%%%%%%%%%%%%%%%%%%%%%%%
    \fill
    \newpage
    \section{Lessons Learned}
    \label{sec:lessonsLearned}

    \subsection{Analyseergebnisse der Enable-Fragebogenrückläufer}
    \label{subsec:screening-ergebnisse}

    \subsubsection{Ergebnisse SUS}
    \label{subsubsec:screening-ergebnisse-sus}



%%%%%%%%%%%%%%%%%%%%%%%%%%%%%%%%%%%%%%%%%%%%%%%%%%%%%%%%%%%%%%%%%%%%%%%%%%%%%%%%%
%	7.Danksagung
%%%%%%%%%%%%%%%%%%%%%%%%%%%%%%%%%%%%%%%%%%%%%%%%%%%%%%%%%%%%%%%%%%%%%%%%%%%%%%%%%
    \vspace{5cm}
    \hline
    \vspace{1cm}
    \noindent
    \textbf{Danksagung}
    \\
    \\
    Wir bedanken uns bei \textit{Prof. Dr.-Ing. Andreas Mayer} für die Betreuung im Rahmen des Kurses Praktikum sichere Software-Entwicklung an der Hochschule Heilbronn.
    \vspace{1cm}
    \hline
    \vspace{2cm}

%%%%%%%%%%%%%%%%%%%%%%%%%%%%%%%%%%%%%%%%%%%%%%%%%%%%%%%%%%%%%%%%%%%%%%%%%%%%%%%%%
%	7.Bibliographie
%%%%%%%%%%%%%%%%%%%%%%%%%%%%%%%%%%%%%%%%%%%%%%%%%%%%%%%%%%%%%%%%%%%%%%%%%%%%%%%%%
    \fill
    \newpage
    \section{Literaturverzeichnis}
    \label{sec:bibliographie}
    \printbibliography[title=""]

%%%%%%%%%%%%%%%%%%%%%%%%%%%%%%%%%%%%%%%%%%%%%%%%%%%%%%%%%%%%%%%%%%%%%%%%%%%%%%%%%
%	8.Abkürzungsverzeichnis
%%%%%%%%%%%%%%%%%%%%%%%%%%%%%%%%%%%%%%%%%%%%%%%%%%%%%%%%%%%%%%%%%%%%%%%%%%%%%%%%%
    \fill
    \newpage

    \section{Abkürzungsverzeichnis}
    \label{sec:abkuerzungsverzeichnis}
    \begin{acronym}
        \acro{xss}[XSS]{Cross-Site Scripting}
        \acro{ilias}[ILIAS]{Integriertes Lern-, Informations- und Arbeitskooperations-System}
        \acro{owasp}[OWASP]{Open Web Application Security Project}
        Open Web Application Security Project
    \end{acronym}


%%%%%%%%%%%%%%%%%%%%%%%%%%%%%%%%%%%%%%%%%%%%%%%%%%%%%%%%%%%%%%%%%%%%%%%%%%%%%%%%%
%	9.Anhang
%%%%%%%%%%%%%%%%%%%%%%%%%%%%%%%%%%%%%%%%%%%%%%%%%%%%%%%%%%%%%%%%%%%%%%%%%%%%%%%%%
    \fill
    \newpage
    \section{Anhang}
    \label{sec:Anhang}




\end{document}

