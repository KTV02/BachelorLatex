%Preamble
\documentclass[10pt, a4paper,onecolumn ,titlepage]{article}
%\documentclass[man,floatsintext]{apa7}

%Einbindung der Referenzen in das Dokument via BiblaTeX
\usepackage[ngerman]{babel}
\usepackage[style=german,german=quotes]{csquotes}
\usepackage[style=apa,backend=biber]{biblatex}
\DeclareLanguageMapping{american}{american-apa}
\addbibresource{main.bib}

%Packages
\usepackage[utf8]{inputenc}
\usepackage[T1]{fontenc}
\usepackage{amsmath,amssymb,amstext}
\usepackage{graphicx}
\usepackage{xstring}
\usepackage{suffix}
\usepackage[nohyperlinks, printonlyused, withpage, smaller]{acronym}
\usepackage{hyperref}
\usepackage{tabularx}
\usepackage{etoolbox}
\usepackage[nooneline]{caption}
\usepackage{float}
\usepackage{pdfpages}
\usepackage{microtype}
\usepackage{booktabs}
\usepackage{abstract}
\usepackage{tablefootnote}
\usepackage{blkarray}
\usepackage{shorttoc}
\usepackage[shortlabels]{enumitem}
\usepackage{threeparttable}

%Ueberschrieben des kewyowrds befehl
\providecommand{\keywords}[1]
{
    \small
    \textbf{\textit{Keywords --}} #1
}



%Documents
\begin{document}

%Titelseite
    \begin{titlepage}
        \begin{center}

            \vspace*{1cm}

            {\large \textbf{Erstellen eines TryHackMe Raums}}

            \vspace{0.5cm}
            zur Vermittlung von Schachstellen im Zusammenhang mit der Lernplattform ILIAS

            \vspace{1.5cm}

            \textbf{Chris Benz} \\
            \small{chbenz@stud.hs-heilbronn.de}
            \\
            \vspace{0.2cm}
            \textbf{Lennart Kremp}\\
            \small{jkaercher@stud.hs-heilbronn.de}
            \\
            \vspace{0.2cm}
            \textbf{Umut Eke} \\
            \small{sthuerwae@stud.hs-heilbronn.de}
            \\
            \vspace{0.2cm}
            \textbf{Frederik Spieß}\\
            \small{ermeyer@stud.hs-heilbronn.de}


            \vfill

            \begin{figure}[H]
                \centering
                \begin{minipage}[b]{.13\linewidth} % [b] => Ausrichtung an \caption
                    \includegraphics[width=\linewidth]{author_pictures/chris_2}
                \end{minipage}\label{fig:chris}
                \hspace{.005\linewidth}% Abstand zwischen Bilder
                \begin{minipage}[b]{.13\linewidth} % [b] => Ausrichtung an \caption
                    \includegraphics[width=\linewidth]{author_pictures/chris_2}
                \end{minipage}\label{fig:lennart}
                \hspace{.005\linewidth}% Abstand zwischen Bilder
                \begin{minipage}[b]{.13\linewidth} % [b] => Ausrichtung an \caption
                    \includegraphics[width=\linewidth]{author_pictures/chris_2}
                \end{minipage}\label{fig:umut}
                \hspace{.005\linewidth}% Abstand zwischen Bilder
                \begin{minipage}[b]{.13\linewidth} % [b] => Ausrichtung an \caption
                    \includegraphics[width=\linewidth]{author_pictures/chris_2}
                \end{minipage}\label{fig:frederik}
            \end{figure}


            \vfill

            Durchführung im Rahmen der Veranstaltung \\ \glqq Praktikum sichere Software-Entwicklung\grqq{} an der Hochschule Heilbronn und Universität Heidelberg

            \vspace{0.2cm}

            Beaufsichtigt durch Prof. Dr.-Ing. Andreas Mayer

            \vspace{1.0cm}


            \vspace{0.8cm}

            Fakultät für Informatik\\
            Hochschule Heilbronn\\
            \today{}, Heilbronn

        \end{center}

    \end{titlepage}

%%%%%%%%%%%%%%%%%%%%%%%%%%%%%%%%%%%%%%%%%%%%%%%%%%%%%%%%%%%%%%%%%%%%%%%%%%%%%%%%%
%	Abstract
%%%%%%%%%%%%%%%%%%%%%%%%%%%%%%%%%%%%%%%%%%%%%%%%%%%%%%%%%%%%%%%%%%%%%%%%%%%%%%%%%

    \renewcommand*\abstractname{\flushleft\textbf{Abstract}\hfill}
    \begin{abstract}
        \hline
        \vspace{0.5cm}
        \noindent
        \textbf{Hintergrund}: Die fortschreitende Digitalisierung ist in allen Bereichen des Lebens angekommen.
        Apps im Allgemeinen, aber auch neue Entwicklungsfelder wie mHealth-Anwendungen erfahren einen starken Anstieg ihrer Nutzerzahlen.
        Dabei ist eine hohe Usability einer der essenziellen Faktoren erfolgreicher Apps.
        Eine fundierte und frühzeitige Beurteilung der Usability einer App bietet den Entwickler*innen die Möglichkeit, Schwachstellen schnell zu beseitigen.
        Es existiert eine Vielzahl verschiedener Fragebögen zur Evaluierung der Usability.
        Oft verwendete Fragebögen sind der System Usability Scale (SUS) sowie der speziell für mHealth-Apps entwickelte mHealth App Usability Questionnaire (MAUQ).
        Neue Daten der am Universitätsklinikum durchgeführten Enable Studie deuten auf Schwierigkeiten beim Verständnis der Items von SUS und MAUQ hin.

        \vspace{0.5cm}
        \noindent
        \textbf{Zielsetzung}: Das Ziel dieses Projekts ist es, festzustellen, ob es bei den Items der Fragebögen SUS und MAUQ Verständnisprobleme gibt.
        Bestätigt sich diese Hypothese, sind auffällige Items zu identifizieren und abschließend jeweils eine überarbeitete Version der Fragebögen zu entwickeln.

        \vspace{0.5cm}
        \noindent
        \textbf{Methoden}: Die in der Enable Studie erhobenen Fragebögen werden in einem ersten Schritt auf Auffälligkeiten und Anmerkungen untersucht,
        um im Anschluss die Items, durch Berechnung der Item Conspicuity Rate (ICR), zu evaluieren.
        In einem zweiten Schritt werden die beiden Fragebögen durch Interviews mit weiteren Proband*innen untersucht, um Informationen über die Ursachen der Verständnisprobleme zu erhalten.
        Hierbei liegt der Fokus auf den, durch die \ac{icr}, als auffällig identifizierten Items.
        Abschließend werden neue Versionen des SUS und MAUQ entworfen.


        \vspace{0.5cm}
        \noindent
        \textbf{Ergebnisse}: Insgesamt wurden 235 MAUQ Fragebögen und 126 SUS Fragebögen einem Screening unterzogen.
        Im Anschluss wurden 37 Proband*innen für eine weitere Erhebung konsultiert.
        Einige Items der beiden Fragebögen sind aufgrund einer Vielzahl an Auffälligkeiten sowie eindeutiger Bemerkungen der Proband*innen, als auffällig einzustufen. %vielleicht auffällig
        Abschließend wurden auf Grundlage der neu gewonnen Erkenntnisse neue Versionen für SUS und MAUQ entwickelt.


        \vspace{0.5cm}
        \hline
        \vspace{3cm}
        \noindent
        \keywords{Schwachstellen, Ilias, ImageMagic, Privilage Escalation in Linux, XSS}
    \end{abstract}

%%%%%%%%%%%%%%%%%%%%%%%%%%%%%%%%%%%%%%%%%%%%%%%%%%%%%%%%%%%%%%%%%%%%%%%%%%%%%%%%%
%	1.Inhaltsverzeichnis
%%%%%%%%%%%%%%%%%%%%%%%%%%%%%%%%%%%%%%%%%%%%%%%%%%%%%%%%%%%%%%%%%%%%%%%%%%%%%%%%%


    \shorttoc{Inhaltsübersicht}{1}
    \pagebreak
    \tableofcontents
    \vfill
    \pagebreak


%%%%%%%%%%%%%%%%%%%%%%%%%%%%%%%%%%%%%%%%%%%%%%%%%%%%%%%%%%%%%%%%%%%%%%%%%%%%%%%%%
%	1.Einleitung
%%%%%%%%%%%%%%%%%%%%%%%%%%%%%%%%%%%%%%%%%%%%%%%%%%%%%%%%%%%%%%%%%%%%%%%%%%%%%%%%%
    \fill
    \newpage
    \section{Einleitung}
    \label{sec:einleitung}

    \subsection{Gegenstand und Motivation}
    \label{subsec:gegenstand-motivation}
    Mit der zunehmenden Digitalisierung der Gesellschaft steigt die Nutzung verschiedenster Anwendungen in allen Lebensbereichen.
    Auch mHealth-Anwendungen weisen eine steigende Beliebtheit auf und sind im Anbetracht mangelnder Ressourcen im Gesundheitssektor ein effektives Werkzeug zur besseren Patientenversorgung.
    Sie sind Teil des klinischen Alltags, helfen Nutzer*innen ein besseres Empfinden für ihre Gesundheit zu erhalten und vernetzen ärztliches Fachpersonal und Patient*in.
    Eine Reihe durchgeführten Studien konnten zeigen, dass gut konzipiere mHealth-Anwendungen Patient*innen nachhaltig unterstützen und die Kosten im Bereich der Gesundheitsversorgung senken können~\parencite{RN20,RN21,RN22}.
    Trotz aller Vorteile die mHealth-Anwendungen mit sich bringen, wird ein Großteil der entwickelten Anwendungen nicht ausgiebig genutzt, da Fachkräfte aufgrund unzureichender Bewertung der Usability häufig nicht von der Zuverlässigkeit der Systeme überzeugt sind~\parencite{RN19}.
    Es zeigt sich, einer der Hauptaspekte erfolgreicher Anwendungen, ist eine verständliche und intuitive Usability.
    Für die Erhebung der Usability gibt es eine große Auswahl verschiedenster Usability-Fragebögen.
    Um möglichst aussagekräftige und korrekte Informationen über die Usability zu erhalten, ist die Qualität des Fragebogens von Bedeutung.
    Es besteht ein Interesse daran, anerkannte Fragebögen auf Auffälligkeiten zu analysieren und diese gegebenenfalls durch überarbeitete Versionen zu adressieren.

    \subsection{Problemstellung}
    \label{subsec:problemstellung}
    Oft verwendete Usability-Fragebögen sind der \ac{sus}~\parencite{RN8} sowie der speziell für mHealth-Apps entwickelte \ac{mauq}~\parencite{RN14}.
    Das aus den Fragebögen gewonnene Feedback wird meist verwendet, um Änderungen an Systemen vorzunehmen, die die Usability verbessern.
    Je genauer die Fragebögen das Feedback der Benutzer*innen zur Usability widerspiegeln, desto zielgerichteter kann die Usability an deren Bedürfnisse angepasst werden.
    Erste Usability-Fragebogenrückläufer einer am Universitätsklinikum Heidelberg durchgeführten Studie lassen die Hypothese aufkommen, dass die Beantwortung der Fragebögen \ac{sus} und \ac{mauq} den Proband*innen Schwierigkeiten bereitet.
    Dies kann sich auf die Qualität der Ergebnisse auswirken und zu fehlerhaften Rückschlüssen in Bezug auf die Usability eines Systems führen.


    \subsection{Zielsetzung}
    \label{subsec:zielsetzung}
    Die Arbeit ist in drei Unterziele einzuteilen.
    Allen voran gilt es, die zuvor erläuterte Hypothese, dass die Items der Fragebögen \ac{sus} und \ac{mauq} Verständnisprobleme aufweisen, zu bestätigen.
    In Abhängigkeit davon sind, nach der Annahme der Hypothese, die Items der beiden Fragebögen weitergehend zu untersuchen.
    So können potenziell problembehaftete Items herausgefiltert und die Ursachen der Verständnisprobleme beschrieben werden.
    Die herausgearbeiteten Probleme bezüglich der auffälligen Items sind in der Form zu adressieren, dass für beide Fragebögen eine überarbeitete Version zur Verfügung steht.



    \subsection{Frage-/Aufgabenstellungen}
    \label{subsec:frage-aufgabenstellung}
    Zur Bestätigung der anfänglichen, in der Zielsetzung beschriebenen Hypothese, ist ein Analyseplan zu erarbeiten.
    Dieser beschreibt das systematische Vorgehen, nach dem die Fragebogenrückläufer der Enable-Studie zu screenen und die verschiedenen Auffälligkeiten zu erfassen sind.
    Die Auffälligkeiten sind zudem in verschiedene Kategorien einzuteilen und durch Proband*innen verfasste Notizen gesondert zu analysieren.
    Dies soll Aufschluss über das Vorhandensein von Verständnisproblemen und erste Grundlagen für das Herausfiltern auffälliger Items liefern.
    Mittels eines geeigneten Werkzeugs (\ac{icr}) und der gesammelten Informationen sind auffällige Items zu identifizieren.
    Es sind Interviews durchzuführen, die Informationen zu den Ursachen der Verständnisprobleme liefern sollen.
    Dabei kommt den durch Proband*innen während den Interviews getätigten Aussagen eine besondere Bedeutung zu.
    Um eine Vergleichbarkeit zwischen den Interviews zu gewährleisten, ist ein Interviewleitfaden zu entwickeln.
    Mit den erarbeiteten Erkenntnissen sind die auffälligen Items zu überarbeiten und neue Versionen der Fragebögen \ac{sus} und \ac{mauq} zu entwerfen.




%%%%%%%%%%%%%%%%%%%%%%%%%%%%%%%%%%%%%%%%%%%%%%%%%%%%%%%%%%%%%%%%%%%%%%%%%%%%%%%%%
%	2.Grundlagen
%%%%%%%%%%%%%%%%%%%%%%%%%%%%%%%%%%%%%%%%%%%%%%%%%%%%%%%%%%%%%%%%%%%%%%%%%%%%%%%%%
    \fill
    \newpage
    \section{Grundlagen}
    \label{sec:grundlagen}

    \subsection{mHealth-Anwendungen}
    \label{subsec:mhealthanwendungen}











%%%%%%%%%%%%%%%%%%%%%%%%%%%%%%%%%%%%%%%%%%%%%%%%%%%%%%%%%%%%%%%%%%%%%%%%%%%%%%%%%
%	3.Methoden und Werkzeuge
%%%%%%%%%%%%%%%%%%%%%%%%%%%%%%%%%%%%%%%%%%%%%%%%%%%%%%%%%%%%%%%%%%%%%%%%%%%%%%%%%
    \fill
    \newpage

    \section{Methoden und Werkzeuge}
    \label{sec:methoden-werkzeuge}

    \subsection{Prozess Übersicht}
    \label{subsec:process-overview}
    Um jeweils einen neuen Entwurf für die Fragebögen SUS und MAUQ zu erhalten, werden zunächst die während der Enable Studie am Universitätsklinikum Heidelberg
    erhobenen Fragebogenrückläufer ausgewertet.
    Die erhobenen Daten bilden die Grundlage für eine Auswahl \textit{auffälliger Items}.
    In einem anschließenden Interview werden weitere Daten mit Fokus auf die, als auffällig klassifizierten, Items erhoben,
    um im Anschluss jeweils einen sprachlich angepassten Entwurf für SUS und MAUQ auszuarbeiten.
    Abbildung ~\ref{fig:flow-chart-übersicht} visualisiert die gewählte Vorgehensweise und stellt einen kompakten Überblick dar.





%%%%%%%%%%%%%%%%%%%%%%%%%%%%%%%%%%%%%%%%%%%%%%%%%%%%%%%%%%%%%%%%%%%%%%%%%%%%%%%%%
%	4.Ergebnisse
%%%%%%%%%%%%%%%%%%%%%%%%%%%%%%%%%%%%%%%%%%%%%%%%%%%%%%%%%%%%%%%%%%%%%%%%%%%%%%%%%
    \fill
    \newpage
    \section{Ergebnisse}
    \label{sec:ergebnisse}

    \subsection{Analyseergebnisse der Enable-Fragebogenrückläufer}
    \label{subsec:screening-ergebnisse}

    \subsubsection{Ergebnisse SUS}
    \label{subsubsec:screening-ergebnisse-sus}







%%%%%%%%%%%%%%%%%%%%%%%%%%%%%%%%%%%%%%%%%%%%%%%%%%%%%%%%%%%%%%%%%%%%%%%%%%%%%%%%%
%	5.Diskussion
%%%%%%%%%%%%%%%%%%%%%%%%%%%%%%%%%%%%%%%%%%%%%%%%%%%%%%%%%%%%%%%%%%%%%%%%%%%%%%%%%
    \fill
    \newpage
    \section{Diskussion}
    \label{sec:diskussion}

    \subsection{MAUQ Item 17 - ein Sonderfall}
    \label{subsec:mauq-item-17-diskussion}
    \noindent
    Item 17 des MAUQ Fragebogens stellt in seiner Komplexität ein Sonderfall dar.
    Es verlangt von den Proband*innen technisches Wissen im Bereich der Funktionsweise von Apps.
    Die in Abschnitt~\ref{subsubsec:interview-ergebnisse-mauq} durch Proband*innen angemerkten Schwierigkeiten bei der Beantwortung können nur zum Teil adressiert werden.
    Im Rahmen eines einzelnen Items ist es nicht möglich, Proband*innen mit mangelndem technischen Verständnis darüber aufzuklären, dass Apps für einen bestimmten Funktionsumfang auf eine intakte Internetverbindung zurückgreifen müssen.
    Es ist davon auszugehen, dass Item 17 in Abgrenzung zu den anderen Items häufiger leer gelassen oder durch ein Fragezeichen markiert sein wird.
    Ein Umstand, der bei einer solch technischen Frage, nicht zu vermeiden ist.





%%%%%%%%%%%%%%%%%%%%%%%%%%%%%%%%%%%%%%%%%%%%%%%%%%%%%%%%%%%%%%%%%%%%%%%%%%%%%%%%%
%	6.Ausblick
%%%%%%%%%%%%%%%%%%%%%%%%%%%%%%%%%%%%%%%%%%%%%%%%%%%%%%%%%%%%%%%%%%%%%%%%%%%%%%%%%
    \fill
    \newpage
    \section{Ausblick}
    \label{sec:ausblick}

    \subsection{Anpassung der sprachlichen Formulierung an die Nutzungszeit der App}
    \label{subsec:konjunktiv}
    Die Auswertungen der Interviews zeigen, dass die gelegentliche Verwendung des Konjunktivs für Proband*innen unerwartet ist.
    Die Verwendung des Modus verursacht bei den Proband*innen jedoch keine Verständnisprobleme.
    Dennoch ist es sinnvoll zwei Versionen der Fragebögen zu entwickeln und validieren.
    Diese sollten die Nutzungsdauer, die den Proband*innen zum Zeitpunkt der Befragung zur Verfügung stand, durch die Verwendung oder den Verzicht des Konjunktivs miteinbeziehen.
    Ab wann der Nutzungszeitraum eine App lange genug ist, um auf die Verwendung des Konjunktivs zu verzichten, gilt es zu ermitteln.
    Hierdurch findet eine Anpassung an den im Alltag genutzten Sprachgebrauch statt, welche voraussichtlich eine verbesserte Lesbarkeit zum Resultat hat.




%%%%%%%%%%%%%%%%%%%%%%%%%%%%%%%%%%%%%%%%%%%%%%%%%%%%%%%%%%%%%%%%%%%%%%%%%%%%%%%%%
%	7.Danksagung
%%%%%%%%%%%%%%%%%%%%%%%%%%%%%%%%%%%%%%%%%%%%%%%%%%%%%%%%%%%%%%%%%%%%%%%%%%%%%%%%%
    \vspace{5cm}
    \hline
    \vspace{1cm}
    \noindent
    \textbf{Danksagung}
    \\
    \\
    Wir bedanken uns bei \textit{Prof. Dr.-Ing. Andreas Mayer} für die Betreuung im Rahmen des Kurses Praktikum sichere Software-Entwicklung an der Hochschule Heilbronn.
    \vspace{1cm}
    \hline
    \vspace{2cm}

%%%%%%%%%%%%%%%%%%%%%%%%%%%%%%%%%%%%%%%%%%%%%%%%%%%%%%%%%%%%%%%%%%%%%%%%%%%%%%%%%
%	7.Bibliographie
%%%%%%%%%%%%%%%%%%%%%%%%%%%%%%%%%%%%%%%%%%%%%%%%%%%%%%%%%%%%%%%%%%%%%%%%%%%%%%%%%
    \fill
    \newpage
    \section{Literaturverzeichnis}
    \label{sec:bibliographie}
    \printbibliography[title=""]

%%%%%%%%%%%%%%%%%%%%%%%%%%%%%%%%%%%%%%%%%%%%%%%%%%%%%%%%%%%%%%%%%%%%%%%%%%%%%%%%%
%	8.Abkürzungsverzeichnis
%%%%%%%%%%%%%%%%%%%%%%%%%%%%%%%%%%%%%%%%%%%%%%%%%%%%%%%%%%%%%%%%%%%%%%%%%%%%%%%%%
    \fill
    \newpage

    \section{Abkürzungsverzeichnis}
    \label{sec:abkuerzungsverzeichnis}
    \begin{acronym}
        \acro{UX}[UX-Design]{User Experience Design}
    \end{acronym}


%%%%%%%%%%%%%%%%%%%%%%%%%%%%%%%%%%%%%%%%%%%%%%%%%%%%%%%%%%%%%%%%%%%%%%%%%%%%%%%%%
%	9.Anhang
%%%%%%%%%%%%%%%%%%%%%%%%%%%%%%%%%%%%%%%%%%%%%%%%%%%%%%%%%%%%%%%%%%%%%%%%%%%%%%%%%
    \fill
    \newpage
    \section{Anhang}
    \label{sec:Anhang}




\end{document}

